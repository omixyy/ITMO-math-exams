\documentclass[14pt]{extreport}
\usepackage{graphicx}
\usepackage{gost}
\usepackage[T1,T2A]{fontenc}
\usepackage[utf8]{inputenc}
\PassOptionsToPackage{russian,english}{babel}
\usepackage[english,russian]{babel}
\usepackage{tempora}
\usepackage{hyperref}
\linespread{1.3}
\setlength{\parindent}{1.25cm}
\usepackage{hyperref}

\usepackage{multirow}
\usepackage{makecell}
\usepackage{longtable}
\usepackage{array}
\newcolumntype{P}[1]{>{\raggedright\arraybackslash}p{#1}}

\usepackage{fancyhdr}
\pagestyle{fancy}
\fancyhf{} 
\fancyhead{} 
\fancyfoot[C]{\thepage} 

\usepackage{amsmath, amssymb}
\usepackage{tcolorbox}
\usepackage{enumitem}
\usepackage{geometry}
\usepackage{amsthm}

\tcbuselibrary{breakable}

\newtcolorbox{thbox}[1][]{colback=orange!10, colframe=orange!60, title=Теорема, #1, breakable=true, toprule at break=0pt, % верхняя рамка не показывается при разрыве
    bottomrule at break=0pt}
\newtcolorbox{prbox}[1][]{colback=green!5, colframe=green!60!gray, title=Доказательство, breakable=true, #1, toprule at break=0pt, % верхняя рамка не показывается при разрыве
    bottomrule at break=0pt}
\newtcolorbox{defbox}[1][]{colback=blue!5, colframe=blue!60, title=Определение, #1, breakable=true, toprule at break=0pt, % верхняя рамка не показывается при разрыве
    bottomrule at break=0pt}
\newtcolorbox{impbox}[1][]{colback=red!5, colframe=red!60, title=Важно, #1, breakable=true, toprule at break=0pt, % верхняя рамка не показывается при разрыве
    bottomrule at break=0pt}

\newtcolorbox{rulebox}[1][]{colback=purple!5, colframe=purple!60, title=Правило, #1, breakable=true, toprule at break=0pt, % верхняя рамка не показывается при разрыве
    bottomrule at break=0pt}

\newcommand{\defeq}[0]{\stackrel{\text{def}}{=}}

\begin{document}
\thispagestyle{empty}
	\title{Дискретная Математика. Экзамен.}
	\title{Дискретная Математика. Экзамен.}
	\maketitle

	\newpage
	\tableofcontents
	
	\chapter{Способы задания множества. Порождающая процедура. Характеристическое свойство.}

	\section{Способы задания множества}

	\begin{enumerate}
		\item Перечисление $\lbrace a, b, c ... \rbrace$
		\item Характеристическое свойство $M \stackrel{\text{def}}{=} \lbrace x : P(x) \rbrace$, где $P(x)$ -- предикат
		\item Порождающая процедура $M := \lbrace y : y = f(x), x \in E \rbrace$, где $f$ -- функция от $x$
	\end{enumerate}
	\section{Описание способов задания множества}

	\begin{defbox}
		\textbf{Характеристическое свойство} -- способ задания множества, при котором каждый его элемент обладает свойством $P(x)$

		$$
		M \defeq \lbrace x : P(x) \rbrace
		$$
	\end{defbox}

	\begin{defbox}
		\textbf{Порождающая процедура} -- способ задания множества, при котором каждый его элемент является результатом выполнения функции $f$ от переменной $x$ из некоторого множества $E$.

		$$
		M := \lbrace y : y = f(x), x \in E \rbrace
		$$
	\end{defbox}

	\chapter{Отображения. Инъекция, сюръекция, биекция. Прямые произведения множеств}

	\section{Отображения}

	\begin{defbox}
		Отображение -- способ сопоставления элементов между множествами.
	\end{defbox}

	\noindent Отображения бывают трех видов:
	\begin{enumerate}
		\item Инъекция
		\item Сюръекция
		\item Биекция
	\end{enumerate}

	\begin{defbox}
		\textbf{Инъекция} -- такое отображение $f(A) \to B$, при котором любой элемент $B$ имеет \textbf{не более одного} прообраза в множестве $A$.
	\end{defbox}

	\begin{defbox}
		\textbf{Сюръекция} -- такое отображение $f(A) \to B$, при котором любой элемент $B$ имеет \textbf{не менее одного} прообраза в множестве $A$.
	\end{defbox}

	\begin{defbox}
		\textbf{Биекция} -- это отображение, являющееся и сюръекцией, и инъекцией одновременно. То есть взаимооднозначное соответствие.
	\end{defbox}

	\section{Прямые произведения множеств}

	\begin{defbox}
		\textbf{Прямое (декартово) произведение множеств} $A$ и $B$ -- все такие пары чисел $(a, b)$, где $a \in A$, $b \in B$.

		$$
		A \times B \defeq \lbrace (a, b) : a \in A, b \in B \rbrace
		$$
	\end{defbox}

	\chapter{Мощность множества}
	\begin{defbox}
		\textbf{Мощность множества} -- количество элементов в нем (для конечных множеств). 
		
		Для бесконечных:
		\begin{enumerate}
			\item Если между бесконечным множеством $X$ и множеством натуральных чисел $\mathbb{N}$ существует биекция, то говорят, что $X$ имеет счётную мощность. Это "наименьшая" бесконечная мощность.
			\item Если между $X$ и множеством всех вещественных чисел $\mathbb{R}$ (или отрезком $[0,1]$) существует биекция, то говорят, что $X$ имеет мощность континуума.
		\end{enumerate}
	\end{defbox}

	\chapter{Операции над множествами.}

	\section{Операции}

	\begin{enumerate}
		\item $A\cap B$ -- пересечение 
		$$
		A \cap B \defeq \lbrace x : (x\in A) \wedge (x\in B) \rbrace
		$$
		\item $A\cup B$ -- объединение
		$$
		A\cap B \defeq \lbrace x : (x\in A) \vee (x \in B) \rbrace
		$$
		\item $A \setminus B$ -- разность
		$$
		A \setminus B \defeq \lbrace x : (x \in A) \wedge (x \notin B)\rbrace
		$$
		\item $A\triangle B$ -- симметрическая разность
		$$
		A \setminus B \defeq \lbrace x : (x \in (A \setminus B)) \vee (x \in (B \setminus A))\rbrace
		$$
		\item $\bar A$ -- дополнение до универсума
		$$
		\bar A \defeq \lbrace x : (x \notin A) \wedge (x \in U) \rbrace
		$$

	\end{enumerate}

	\section{Свойства}
	\begin{enumerate}
		\item $A \cap A = A$, $A \cup A = A$
		\item $A \cap (B \cap C) = (A \cap B) \cap C$,
				
				$A \cup (B \cup C) = (A \cup B) \cup C$
		\item $A \cap (B \cup C) = (A \cap B) \cup (A \cap C)$, 
		
				$A \cup (B \cap C) = (A \cup B) \cap (A \cup C)$ 

		\item $A \cup \emptyset = A$, $A \cap \emptyset = \emptyset$
		\item $A \cup U = U$, $A \cap U = A$
		\item $A \cap B = B \cap A$, $A \cup B = B \cup A$
		\item $A \cup B = A \cap \bar{B}$,  
		
				$\overline{A \cap B} = \bar{A} \cup \bar B$
		\item $\bar{ \bar A} = A$
		\item $A \setminus B = A \cap \bar B$
	\end{enumerate}

	\chapter{Прямое произведение $N$ множеств.}

	\begin{defbox}
		\textbf{Прямое произведение $n$ множеств} -- все возможные кортежи из элементов этих $n$ множеств.

		На примере двух множеств:
		$$
		A \times B = \lbrace (a, b) : a\in A, b \in B \rbrace
		$$
	\end{defbox}

	\chapter{Теорема о мощности прямого произведения множеств}

	\begin{thbox}
		Если $A$ и $B$ конечны, $|A| = n$, $|B| = m$ то $|A \times B| = |A| |B| = mn$
	\end{thbox}

	\begin{prbox}
		Рассмотрим кортеж. В нем на первом месте стоит элемент из $A$, на втором -- из $B$. К каждому элементу из $A$ можно приставить $m$ элементов из $B$, получив тем самым множество, представляющее собой результат декартового произведения. То есть на первом месте в кортеже может стоять $n$ элементов, на втором -- $m$. Значит всего таких кортежей можно составить $mn$ штук.
	\end{prbox}

	\chapter{Понятие вектора. Проекция вектора на оси. Проекция множества векторов}

	\begin{defbox}
		Пусть задано прямое произведение $A_1 \times A_2 \times ... \times A_n$. \textbf{Вектором} называется упорядоченный набор $(a_1, a_2, ... a_n)$, где $a_i \in A_i$
	\end{defbox}

	\begin{defbox}
		Пусть задано прямое произведение $A_1 \times A_2 \times ... \times A_n$. \textbf{Проецированием $\operatorname{Pr}_k(a_1, a_2, ... a_k...a_n)$} называется отображение $(a_1, a_2, ... a_n) \rightarrow (a_k)$
		$$
		\operatorname{Pr}_k(a_1, a_2, ... a_k...a_n) \defeq f((a_1, a_2, ... a_k...a_n)) \rightarrow (a_k)
		$$
		Проецирование множества векторов:
		$$
		\operatorname{Pr}_{i, j, ... m}(\lbrace (a_1, a_2,...a_m), (b_1, b_2,...b_m)... \rbrace) = 
		$$
		$$
		= \lbrace (a_i, a_j,...a_m), (b_i, b_j, ... b_m)... \rbrace
		$$
	\end{defbox}

	\chapter{Правило суммы. Правило произведения.}

	\begin{thbox}
		{[Правило суммы]}

		Пусть все выборки из множества $A$ делятся на две взаимоисключающие $A_1$ и $A_2$. Число выборок первого типа $m_1$, второго -- $m_2$. Тогда число всех выборок из множества $A$ равно $m_1 + m_2$
	\end{thbox}

	\begin{prbox}
		Данное правило является следствием формулы включений-исключений:
		$$
		|A \cup B| = |A| + |B| - |A \cap B|
		$$
		где $A \cap B = \emptyset$
	\end{prbox}

	\begin{thbox}
		{[Правило произведения]}
		
		Пусть число способов построить выборку из множества $A$ равно $n$, из множества $B$ -- $m$. Тогда число способов построить выборку $(a, b)$ ($a \in A, b \in B$) равно $mn$
	\end{thbox}

	\begin{prbox}
		Данное утверждение эквивалентно теореме о мощности прямого произведения. $|A \times B| = |A| |B|=mn$
	\end{prbox}

	\chapter{Число размещений без повторений, число размещений с повторениями}

	\begin{defbox}
		Пусть имеется множество из $n$ элементов. Упорядоченное подмножество из $k$ элементов называется размещением без повторений
		
	\end{defbox}
	\begin{thbox}
		Число размещений без повторений можно рассчитать по формуле:
		$$
		A_n^k = \frac{n!}{(n-k)!}
		$$
	\end{thbox}

	\begin{prbox}
		Пусть есть $n$ элементов, из которых нужно составить упорядоченные наборы из $k$ элементов.

		Тогда на первое место можно поставить $n$ элементов, на второе -- $n-1$, на третье -- $n-2$ и так далее до $k$-го места, куда можно поставить $n-k+1$ элементов. тогда посчитаем общее количество наборов по правилу произведения: 
		$$
		n \cdot (n-1) \cdot (n-2) ... (n-k+1) = \frac{n!}{(n-k)!}
		$$
	\end{prbox}

	\vspace{20pt}
	\begin{defbox}
		Пусть имеется множество $A$ из $n$ элементов. Набор $(m_1, m_2 ... m_k)$, где $\forall i \Rightarrow m_i \in A$, называется размещением с повторениями.
	\end{defbox}

	\begin{thbox}
		Число размещений с повторениями можно рассчитать по формуле:
		$$
		\overline{A_n^k} = n^k
		$$
	\end{thbox}

	\begin{prbox}
		Есть $k$ позиция, на каждой тз них $n$ элементов. Тогда по правилу произведения всего $n^k$ наборов
	\end{prbox}

	\chapter{Число перестановок без повторений, число перестановок с повторениями.}

	\begin{defbox}
		Пусть имеется множество из $n$ элементов. \textbf{Перестановкой с повторениями} называется упорядоченная последовательность его элементов
	\end{defbox}

	\begin{thbox}
		Перестановки без повторений можно посчитать по формуле:
		$$
		P_n = n!
		$$
	\end{thbox}
	\begin{prbox}
		Данная формула является следствием правила произведения. Есть $n$ позиций, на каждой следующей на $1$ элемент меньше, чем на предыдущей. Значит формула:
		$$
		P_k = n!
		$$
	\end{prbox}

	\vspace{20pt}
	\begin{defbox}
		Пусть имеется множество из $n$ элементов, среди которых:
		\begin{itemize}
			\item $k_1$ неразличимых элементов 1-го типа
			\item $k_2$ неразличимых элементов 2-го типа
			
			\dots

			\item $k_s$ неразличимых элементов s-го типа
		\end{itemize}
		\textbf{Перестановкой с повторениями} называется упорядоченная последовательность элементов этого множества 
	\end{defbox}

	\begin{thbox}
		Число перестановок с повторениями можно рассчитать по формуле:
		$$
		\overline{P_n} = \frac{(k_1 + k_2 + ... + k_s)!}{k_1!k_2!...k_s!}
		$$
		где $k_1+k_2+...+k_s=n$
	\end{thbox}
	\begin{prbox}
		Для начала сосчитаем количество перестановок без повторений (представим, что в исходном множестве разные элементы): $n!$. Теперь сосчитаем перестановки для каждой группы: $k_i!$. Поскольку в исходном множестве есть группы одинаковых элементов, то их перестановки для нас неразличимы, а значит необходимо убрать все перестановки, в которых одинаковые наборы элементов одного типа, а их, как мы сосчитали, для каждого типа $k_i!$. Итого формула:
		$$
		\overline{P_n} = \frac{n!}{k_1!k_2!...k_s!}
		$$
	\end{prbox}

	\chapter{Число сочетаний без повторений, число сочетаний с повторениями.}

	\section{Сочетания без повторений}
	\begin{defbox}
		Пусть имеется множество из $n$ элементов. Неупорядоченное подмножество из $k$ его элементов называется \textbf{сочетанием без повторений}
	\end{defbox}

	\begin{thbox}
		Число сочетаний без повторений рассчитывается по формуле:
		$$
		C_n^k = \frac{n!}{(n-k)!k!}
		$$
	\end{thbox}

	\begin{prbox}
		Посмотрим на формулу размещений без повторений: $A_n^k = \frac{n!}{(n-k)!}$. Чтобы рассчитать сочетания, небходимо убрать из размещений все те наборы, в которых совпадают элементы. Для каждого набора элементов таких повторяющихся наборов $k!$, ведь это просто перестановки. Тогда итоговая формула:
		$$
		C_n^k = \frac{n!}{(n-k)!k!}
		$$
	\end{prbox}

	\section{Сочетания с повторениями}
	\begin{defbox}
		Пусть имеется $k$ классов элементов множества $A$. Сочетанием с повторениями называется неупорядоченная выборка $n$ элементов из множества $A$.
	\end{defbox}

	\begin{thbox}
		Число сочетаний с повторениями можно рассчитать по формуле:
		$$
		\overline{C_n^k} = C_{n+k-1}^{k-1}=\frac{(n+k-1)!}{n!(k-1)!}
		$$
	\end{thbox}

	\begin{prbox}
		Используем метод "звезд и перегородок": будем рассматривать $k$ звезд и $n-1$ перегородок для $n$ типов звезд. Любая такая последовательность однозначно задаёт одно сочетание с повторениями, и наоборот -- любому сочетанию соответствует такая последовательность. 

		Общее число символов в последовательности: $n + k - 1$. Из них выберем $k$ позиций для звезд (или $n-1$ для перегородок). Тогда таких последовательностей:
		$$
		\overline{C_n^k} = \frac{(n+k-1)!}{n!(k-1)!}
		$$
	\end{prbox}

	\chapter{Соответствия и функции}
	\section{Соответствия}
	\begin{defbox}
		Пусть даны два множества $A$ и $B$. \textbf{Соответствием} называется подмножество $A\times B$.
	\end{defbox}
	\begin{defbox}
		\textbf{Областью определения} $G$ называется $\operatorname{Pr}_A G$
	\end{defbox}
	\begin{defbox}
		\textbf{Областью значений} $G$ называется $\operatorname{Pr}_B G$
	\end{defbox}
	\begin{defbox}
		\textbf{Образом элемента} $a$ называется множество всех тех элементов $b$, которые входят в пары $(a, b) \in G$
		$$
		G(a) = \lbrace b : (a, b) \in G \rbrace
		$$ 
	\end{defbox}
	\begin{defbox}
		\textbf{Прообразом элемента} $b$ называется множество всех тех элементов $a$, которые входят в пары $(a, b) \in G$
		$$
		G^{-1}(b) = \lbrace a : (a, b) \in G \rbrace
		$$ 
	\end{defbox}
	\begin{defbox}
		Соответствие называется \textbf{полностью определенным}, если его областью определения является множество $A$.
	\end{defbox}
	\begin{defbox}
		Соответствие называется \textbf{сюръективным}, если его областью значений является множество $B$
	\end{defbox}
	\begin{defbox}
		Соответствие называется \textbf{инъективным}, если каждый элемент в области значений имеет ровно один прообраз
	\end{defbox}
	\begin{defbox}
		Соответствие называется \textbf{функциональным}, если каждый элемент в его области определения имеет не более одного образа
	\end{defbox}
	\begin{defbox}
		Соответствие называется \textbf{биективным}, если:
		\begin{enumerate}
			\item Является полностью определенным
			\item Функциональным
			\item инъективным
			\item Сюръективным
		\end{enumerate}
	\end{defbox}
	\begin{impbox}
		Отображение \textbf{"в", если оно не является сюръективным}, и \textbf{"на" в противоположном случае}. 
	\end{impbox}

	\section{Функции}
	\begin{defbox}
		\textbf{Функция} -- это всюду определенное и функциональное соответствие
	\end{defbox}
	\begin{defbox}
		Функция называется \textbf{инъективной}, если разным аргументам соответствуют разные значения
	\end{defbox}
	\begin{defbox}
		Функция называется \textbf{сюръективной}, если все элементы множества значений "покрыты"
	\end{defbox}
	\begin{defbox}
		Функция называется \textbf{биективной}, если она является сюръективной и инъективной.
	\end{defbox}
	\begin{defbox}
		Для любого соответствия $R \subseteq A \times B$ можно определить \textbf{обратное соответствие} $R^{-1} \subseteq B \times A$
	\end{defbox}
	\begin{impbox}
		Обратное соответствие будет функцией тогда и только тогда, когда функция биективна
	\end{impbox}

	\chapter{Взаимно однозначные соответствия и мощность множеств}

	\begin{defbox}
		Соответствие называется \textbf{взаимно-однозначным}, если оно сюръективно и инъективно (каждому элементу $A$ сопоставляется ровно один элемент $B$)
	\end{defbox}
	\section{Установление взаимно-однозначного соответствия между множествами чисел}

	\begin{thbox}
		Множество $\mathbb{Z}$ счетно
	\end{thbox}
	\begin{prbox}
		Попробуем установить биекцию между $\mathbb{N}$ и $\mathbb{Z}$. Действительно, если установить соответствие следующим образом ($f: \mathbb{N} \rightarrow \mathbb{Z}$):
		$$
		f(2n) = n
		$$
		$$
		f(2n+1) = -n
		$$
		$$
		f(1) = 0
		$$
		То окажется, что каждому элементу $\mathbb{N}$ сопоставлен ровно один элемент $\mathbb{Z}$ и наоборот, а значит получена биекция.

		Множество $\mathbb{Z}$ счетно.
	\end{prbox}
	\vspace{20pt}
	\begin{thbox}
		Множество $\mathbb{Q}$ счетно
	\end{thbox}
	\begin{prbox}
		Составим таблицу:

		\vspace{10pt}
		\begin{tabular}{|c|c|c|c|c|}
			\hline
			0 & 1 & -1 & 2 & -2\\
			\hline
			$\frac{0}{2}$ & $\frac{1}{2}$ & $\frac{-1}{2}$ & $\frac{2}{2}$ & $\frac{-2}{2}$\\
			\hline
			$\frac{0}{3}$ & $\frac{1}{3}$ & $\frac{-1}{3}$ & $\frac{2}{3}$ & $\frac{-2}{3}$\\
			\hline
			... & & & & \\
			\hline
		\end{tabular}

		\vspace{10pt}
		В этой таблице представлены все рациональные числа. Теперь можно пойти "змейкой" и сосчитать все эти числа.

		Значит $|\mathbb{Q}| = |\mathbb{N}|$
	\end{prbox}

	\chapter{Теорема о числе подмножеств конечного множества.}
	\begin{thbox}
		Число всех подмножеств конечного множества, состоящего из $n$ элементов, равно $2^n$
	\end{thbox}
	\begin{prbox}
		Каждое подмножество исходного множество можно задать, установив "флаг" каждому его элементу: включать его в подмножество или нет. Всего два варианта, а значит по правилу произведения всего вариантов подмножеств: 
		$$
		2 \cdot 2 \cdot ... \cdot 2 = 2 ^ n
		$$
	\end{prbox}

	\chapter{Число подмножеств счетного множества.}
	\begin{thbox}
		Множество всех подмножеств счетного множества несчетно
	\end{thbox}
	\begin{prbox}
		Предположим, что все подмножества данного счетного множества можно пронумеровать: $A_1, A_2, ... A_n$ и построим множество $D$, противоречащее допущению, по следующему правилу:

		\textit{Число $n$ входит в $D$ тогда и только тогда, когда оно не входит в $A_n$}

		$$
		D = \lbrace n \in \mathbb{N} : n \notin A_n \rbrace
		$$

		Но тогда получим противоречие: $D$ -- корректно построенное множество натуральных чисел, а значит оно должно являться подмножеством $\mathbb{N}$

		Значит $|\mathbb{N}| \ne |2^\mathbb{N}|$
	\end{prbox}

		\chapter{Теорема Кантора}
	\begin{thbox}
		{[Теорема Кантора]}

		Мощность множества всегда меньше мощности множества всех его подмножеств

		$$
		|A| < |2^A|
		$$
	\end{thbox}

	\begin{prbox}
		Очевидно, что $|A| \le |2^A|$, поскольку $2^A$ содержит в качестве подмножеств все элементы $A$, а также другие подмножества, сочетающие эти элементы, и пустое множество.

		Докажем, что $|A| \ne |2^A|$

		Зададим множество $B$ следующим образом:
		$$
		B = \lbrace x\in A : x \notin f(x) \rbrace
		$$
		То есть это множество всех тех $x$ из $A$, не принадлежащих своим прообразам при биективном отображении $f$.

		Понятно, что $B \in 2^A$, ведь $B$ состоит из элементов $A$. Попробуем найти прообраз $B$ в множестве $A$.

		Если существует $x_0 \in A$ такое, что $f(x_0) = B$, то получаем противоречие, ведь $B$ составлено из элементов множества $A$, которых нет в их образе в $2^A$. Значит такого $x_0$ не существует. Но это означает, что у $B$ не существует прообраза в $A$, а значит не существует и биекции между $A$ и $2^A$.
	\end{prbox}

	\chapter{Графы. Определение графа. Ориентированные и неориентированные графы.}

	\begin{defbox}
		\textbf{Граф} -- это отношения инцидентности, заданные на множестве вершин и ребер
		$$
		\Gamma : e \rightarrow (u; v)
		$$
	\end{defbox}
	\begin{defbox}
		Вершина и ребро называются \textbf{инцидентными}, если вершина является концом ребра.
	\end{defbox}
	\begin{defbox}
		Граф называется \textbf{ориентированным}, если в нем упорядочены вершины
	\end{defbox}

	\chapter{Матрица инцидентности.}
	\begin{defbox}
		\textbf{Матрица инцидентности }-- это матрица, в которой по горизонтали расположены вершины, по вертикали -- ребра.

		В неориентированном графе:
		$$
		e_{ij} = \begin{cases}
			1, \text{если ребро инцидентно вершине}\\
			2, \text{иначе}
		\end{cases}
		$$

		В ориентированном графе:
		$$
		e_{ij} = \begin{cases}
			1, \text{если вершина $j$ -- конец ребра $i$}\\
			0, \text{если неинцидентны}\\
			-1, \text{если вершина $j$ -- начало ребра $i$}
		\end{cases}
		$$
	\end{defbox}

	\chapter{Матрица смежности.}
	\begin{defbox}
		\textbf{Матрица смежности }-- матрица, в которой по горизонтали и вертикали -- вершины
		$$
		e_{ij} = \begin{cases}
			1, \text{если вершины смежны}\\
			-, \text{иначе}
		\end{cases}
		$$
	\end{defbox}

	\chapter{Локальная степень вершины. Вектор локальных степеней.}
	\begin{defbox}
		\textbf{Степень вершины} -- это количество ребер, инцидентных вершине
	\end{defbox}

	\begin{defbox}
		\textbf{Вектор степеней} -- вектор, составленный из степеней всех вершин графа, расположенных по убыванию
	\end{defbox}


	\chapter{Маршрут общего вида. Простой путь и не простой путь.}
	\section{Маршрут общего вида}
	\begin{defbox}
		Маршрут — это самая общая 
		последовательность вершин и 
		ребер, в которой каждое ребро 
		соединяет две соседние вершины 
		в последовательности.
	\end{defbox}
	\section{путь}
	\begin{defbox}
		Простой путь - Это маршрут, в котором все вершины (и, соответственно, ребра) различны.
	\end{defbox}
	\begin{defbox}
		Непростой путь - Это маршрут, в котором ребра не повторяются, но вершины могут повторяться.
	\end{defbox}
	\section{Замкнутые цепи}
	\begin{itemize}
		\item Цикл: Замкнутая цепь (ребра не повторяются).
		\item Простой цикл: Замкнутый простой путь (вершины не повторяются, кроме начальной и конечной).
	\end{itemize}

	\chapter{Циклический маршрут общего вида}
	\begin{defbox}
		Циклический маршрут — это маршрут $v_0, e_1, v_1, e_2, \dots, e_n, v_n$, 
		у которого начальная и конечная вершины совпадают ($v_0 = v_n$).
		Главная особенность маршрута общего вида 
		заключается в отсутствии жестких ограничений:
		\begin{itemize}
			\item В нем могут многократно повторяться ребра.
			\item В нем могут многократно повторяться вершины.
			\item Он может содержать внутри себя другие циклы
		\end{itemize}
	\end{defbox}

	\chapter{Простой и не простой цикл.}
	\section{Простой цикл}
	\begin{defbox}
		Простой цикл — это замкнутый путь, в котором все вершины и все ребра различны
	\end{defbox}

	\section{Не простой цикл}
	\begin{defbox}
		Не простой цикл — это замкнутый маршрут, в котором ребра не повторяются, но вершины могут повторяться.
	\end{defbox}

	\chapter{Матрица расстояний. Эксцентриситет вершины. Диаметр графа. Радиус графа. Центр
	графа.
	}
	\section{Матрица расстояний}
	\begin{defbox}
		Матрица расстояний $D$ — это квадратная таблица размера $n \times n$ (где $n$ — число вершин), 
		в которой элемент $d_{ij}$ равен кратчайшему расстоянию (количеству ребер) между вершинами $v_i$ и $v_j$.
		\begin{itemize}
			\item Если пути между вершинами нет, расстояние считается бесконечным ($\infty$).
			\item На главной диагонали всегда стоят нули ($d_{ii} = 0$).
		\end{itemize}
	\end{defbox}
	\section{Эксцентриситет вершины}
	\begin{defbox}
		Эксцентриситет вершины $v$ — это максимальное расстояние 
		от этой вершины до любой другой вершины графа.$$\epsilon(v) = \max_{u \in V} d(v, u)$$
	\end{defbox}
	\section{Диаметр, Радиус и Центр}
	\begin{defbox}
		\begin{itemize}
			\item Диаметр ($diam(G)$)Это максимальный из всех эксцентриситетов.
			\item Радиус ($rad(G)$)Это минимальный из всех эксцентриситетов.
			\item Центр графа Это множество вершин, у которых эксцентриситет равен радиусу.
		\end{itemize}
	\end{defbox}

	\chapter{Диаметральный путь. Радиальный путь.}
	\section{Диаметральный путь}
	\begin{defbox}
		Диаметральный путь — это кратчайший путь между двумя 
		вершинами графа, расстояние между которыми равно диаметру ($diam(G)$).
	\end{defbox}
	\section{Радиальный путь}
	\begin{defbox}
		Радиальный путь — это кратчайший путь, соединяющий 
		центральную вершину с любой вершиной, расстояние 
		до которой равно эксцентриситету этой центральной вершины
	\end{defbox}

	\chapter{Цикломатическое число.}
	\section{Формула вычисления}
	Цикломатическое число обозначается греческой буквой $\nu$ (ню) или $\mu$ (мю) и вычисляется по формуле:$$\nu(G) = m - n + p$$
	Где:
	\begin{itemize}
		\item $m$ — количество ребер графа;
		\item $n$ — количество вершин графа;
		\item $p$ — количество компонент связности (для связного графа $p = 1$).
	\end{itemize}
	\section{Физический и геометрический смысл}
	\begin{enumerate}
		\item Фундаментальная система циклов: Цикломатическое число равно размерности пространства циклов графа. Это значит, что из всех возможных циклов в графе можно выбрать ровно $\nu(G)$ «базисных», через которые можно выразить любой другой цикл.
		\item Связь с деревьями: Если $\nu(G) = 0$, то граф является лесом (не содержит циклов). Если он при этом связный, то это дерево.
		\item Удаление ребер: Это минимальное количество ребер, после удаления которых граф становится ациклическим (деревом), но при этом сохраняет все свои вершины и компоненты связности.
	\end{enumerate}

	\chapter{Число внутренней устойчивости. Число внешней устойчивости.}
	\section{Число внутренней устойчивости ($\alpha(G)$)}
	\begin{defbox}
		Внутренне устойчивое множество 
		(или независимое множество) — 
		это такой набор вершин графа, 
		в котором никакие две вершины 
		не соединены ребром.
	\end{defbox}
	\section{Число внешней устойчивости ($\beta(G)$ или $\gamma(G)$)}
	\begin{defbox}
		Внешне устойчивое множество (или доминирующее множество) — это такой наименьший возможный набор 
		вершин $S$, что любая вершина графа либо входит в $S$, либо соединена хотя бы с одной вершиной из $S$.
	\end{defbox}

	\chapter{Хроматическое число. Хроматический индекс.}
	\section{Хроматическое число ($\chi(G)$)}
	\begin{defbox}
		Хроматическое число — это минимальное количество цветов, 
		необходимое для раскраски вершин графа 
		так, чтобы любые две соседние вершины (соединенные ребром) имели разные цвета.
	\end{defbox}
	\section{Хроматический индекс ($\chi'(G)$)}
	\begin{defbox}
		Хроматический индекс  — это минимальное количество цветов, необходимое для 
		раскраски ребер графа так, чтобы любые два ребра, имеющие общую вершину, были окрашены в разные цвета.
	\end{defbox}

	\chapter{Деревья. Теорема о висячей вершине. Число ребер в дереве.}
	\begin{defbox}
		Дерево — это связный граф, который не содержит циклов.
	\end{defbox}
	\section{Теорема о висячей вершине}
	Вершина графа называется висячей (или листом), если её степень равна 1 (то есть из неё выходит только одно ребро).
	\begin{thbox}
		В любом конечном дереве, содержащем более одной вершины (при $n \ge 2$), найдутся как минимум две висячие вершины.
	\end{thbox}
	\begin{prbox}
		Если мы начнем путь из любой вершины и будем идти по ребрам, 
		не возвращаясь назад, то из-за отсутствия циклов и конечности числа вершин мы неизбежно 
		зайдем в «тупик». Этот тупик и будет висячей вершиной. Поскольку путь имеет два конца, таких вершин минимум две.
	\end{prbox}
	\section{Число ребер в дереве}
	\begin{thbox}
		Если дерево содержит $n$ вершин и $m$ ребер, то справедливо равенство:$$m = n - 1$$
		\begin{itemize}
			\item Если в связном графе ребер больше, чем $n-1$, в нем обязательно есть хотя бы один цикл.
			\item Если в связном графе ребер меньше, чем $n-1$, такой граф не может быть связным
		\end{itemize}
	\end{thbox}

	\chapter{Корень дерева. Множество листьев дерева. Высота дерева.}
	\section{Корень дерева}
	\begin{defbox}
		Корень — это выделенная вершина дерева, от которой отсчитывается направление ко всем остальным вершинам.
	\end{defbox}
	\section{Множества листьев}
	\begin{defbox}
		Листья (или висячие вершины) — это вершины корневого дерева, у которых нет потомков.
	\end{defbox}
	\section{Уровни и высота дерева}
	\begin{defbox}
		Высота — это мера «глубины» или вертикального размера дерева. Чтобы её определить, нужно сначала понять уровни. 
		У ровень вершины - это расстояние от корня до вершины. Корень находится на 0 уровне. Дальше остается посчитать 
		расстояние до конечной вершины.
	\end{defbox}


	\chapter{Центры дерева. Длина диаметральной цепи. Максимальный тип вершин. Диаметральный путь дерева.
	}
	\section{Центры дерева}
	\begin{defbox}
		Центром дерева называется подмножество вершин с минимальным значением эксцентриситета
	\end{defbox}
	\section{Диаметральный путь дерева}
	\begin{defbox}
		Диаметральным путем дерева называется простейшая цепь между двумя вершинами, имеющая максимально возможную длину.
	\end{defbox}
	\section{Длина диаметральной цепи}
	\begin{defbox}
		Диаметром дерева $d(G)$ называется длина (количество ребер) его диаметрального пути.
	\end{defbox}
	\section{Максимальный тип вершин}
	\begin{defbox}
		Максимальный тип вершин дерева — это значение максимальной степени среди всех его узлов, обозначаемое как 
		$\Delta(G)$.$$\Delta(G) = \max_{v \in V} \text{deg}(v)$$
	\end{defbox}
	


>>>>>>> local
	\chapter{Определение логической функции. Таблица истинности. Мощность множеств логических функций(32)}

	\section{Определение логической функции}

	\begin{defbox}
		Логическая функция – это функция $f(x_1,x_2,...,x_n)$, 
		где и аргументы, и сама функция принимают значения из множества {0,1}.
	\end{defbox}
	\begin{itemize}
		\item Область определения: Множества всех наборов длины $n$ из нулей и единиц.
		 Всего $2^n$ таких наборов.
		\item Область значений: {0,1}.
	\end{itemize}

	\section{Таблица истинности}

	Таблица истинности является самым простым способом здания функции. 
	Для этого необходимо выписать все возможные комбинации входных переменных 
	и результат функции каждой из них.\\
	Количество строк в таблице всегда равно $2^n$, где n – число переменных. \\ 
	Пример для $n = 2$:\\

	
	\vspace{10pt}
	\begin{tabular}{|c|c|c|}
		\hline
		$x$ & $y$ & $x \wedge y$\\
		\hline
		0 & 0 & 0\\
		\hline
		0 & 1 & 0\\
		\hline
		1 & 0 & 0\\
		\hline
		1 & 1 & 1\\
		\hline
	\end{tabular}

	\section{Мощность множества логической функции}

	\begin{thbox}
		\begin{enumerate}
			\item У нас есть n переменных
			\item Количество наборов вариантов переменных равно $2^n$
			\item Для каждой строки в столбце значений функции мы можем выбрать либо 0, либо 1.
			\item Следовательно, общее количество функций вычисляется как 2 в степени, равной количеству строк.
		\end{enumerate}
		Таким образом получаем следующую формулу: $N = 2^{2^n}$
	\end{thbox}


	\chapter{Таблица функций одной переменной. Функции с фиктивными переменными.(33)}
	\section{Таблица функций одной переменной}
	Исходя из формулы $N = 2^{2^n}$ существует всего 4 функции от одной переменной. 
	Пусть есть переменная x. Далее в таблице представлены все возможные варианты того, что функция 
	может выдать на выходе.\\


	\begin{tabular}{|c|c|c|c|c|}
		\hline
		$x$ & $f_0(x)$ & $f_1(x)$ & $f_2(x)$ & $f_3(x)$ \\
		\hline
		0 & 0 & 0 & 1 & 1 \\
		\hline
		1 & 0 & 1 & 0 & 1 \\
		\hline
	\end{tabular}
	\begin{enumerate}
		\item $f_0(x) = 0$ — Константа «ноль». Что бы мы ни подали на вход, на выходе всегда 0.
		\item $f_1(x) = x$ — Тождественная функция (повторитель). Выдает то же самое, что пришло на вход.
		\item $f_2(x) = \bar{x}$ — Отрицание (инверсия). Меняет 0 на 1 и наоборот.
		\item $f_3(x) = 1$ — Константа «единица».
	\end{enumerate}

	\section{Функции с фиктивными переменными}

	\begin{defbox}
		Переменная $x_i$ называется фиктивной для функции $f(x_1, \dots, x_i, \dots, x_n)$, 
		если значение функции не зависит от того, чему равно $x_i$ (при неизменных остальных переменных).
		$$f(x_1, \dots, 0, \dots, x_n) = f(x_1, \dots, 1, \dots, x_n)$$
	\end{defbox}
	Если переменная не фиктивная, она называется существенной.
	Пример:Представим функцию $f(x, y) = x \wedge (y \vee \bar{y})$.Мы знаем, что $(y \vee \bar{y})$ всегда равно 1. Значит, $f(x, y) = x \wedge 1 = x$.
	Здесь переменная $y$ — фиктивная. Мы можем её «выкинуть», и суть функции не изменится.

	\chapter{Таблица функций с 2-мя переменными. Конъюнкция. Дизъюнкция. Штрих Шеффера, стрелка Пирса. 
	Эквивалентность, сложение по модулю 2, импликация. Функции с
	фиктивными переменными.(34)}

	\section{Общая таблица функций 2-х переменных}
	Для двух переменных существует 4 набора возможных значений переменных и 16 возможных функций. \\
	\begin{tabular}{|c|c|c|c|c|c|}
		\hline
		x & y & Конъюн ($x \wedge y$) & Дизъюн ($x \vee y$) & XOR ($\oplus$) &  Стрелка Пирса ($x \downarrow y$) \\
		\hline
		0 & 0 & 0 & 0 & 0 & 1 \\
		\hline
		0 & 1 & 0 & 1 & 1 & 0 \\
		\hline
		1 & 0 & 0 & 1 & 1 & 0 \\
		\hline
		1 & 1 & 1 & 1 & 0 & 0 \\
		\hline
	\end{tabular}
	\begin{tabular}{|c|c|c|c|c|}
		\hline
		x & y & Эквивалентность ($\leftrightarrow$) & Импликация ($x \to y$) & Штрих Шеффера ($x \mid y$) \\
		\hline
		0 & 0 & 1 & 1 & 1 \\
		\hline
		0 & 1 & 0 & 1 & 1 \\
		\hline
		1 & 0 & 0 & 0 & 1 \\
		\hline
		1 & 1 & 1 & 1 & 0 \\
		\hline
	\end{tabular}

	\section{все функции}
	\begin{itemize}
		\item Конъюнкция ($x \wedge y$): Логическое «И». Истинна только когда оба аргумента — единицы. Похожа на обычное умножение.
		\item Дизъюнкция ($x \vee y$): Логическое «ИЛИ». Истинна, если есть хотя бы одна единица.
		\item Сложение по модулю 2 ($x \oplus y$): Оно же XOR или «исключающее ИЛИ». Истинна, только когда аргументы разные. (Важно: $1 \oplus 1 = 0$).
		\item Эквивалентность ($x \leftrightarrow y$): Наоборот, истинна, когда аргументы одинаковые. Это отрицание XOR.
		\item Импликация ($x \to y$): Логическое следование. Единственный случай, когда она ложна: из истины следует ложь ($1 \to 0 = 0$). В остальных случаях — 1. Запомни: «из лжи может следовать что угодно».
		\item Штрих Шеффера ($x \mid y$): Это «И-НЕ» ($\neg(x \wedge y)$). Ложен только на наборе $(1,1)$.
		\item Стрелка Пирса ($x \downarrow y$): Это «ИЛИ-НЕ» ($\neg(x \vee y)$). Истинна только на наборе $(0,0)$.
	\end{itemize}

	\section{Функции с фиктивными переменными}

	Пусть дана функция $f(x, y)$, где столбец значений выглядит так: 0, 0, 1, 1.
	\begin{enumerate}
		\item Сравниваем строки $f(0, 0)$ и $f(0, 1)$. Оба раза результат 0.
		\item Сравниваем строки $f(1, 0)$ и $f(1, 1)$. Оба раза результат 1.
		Вывод: Изменение $y$ ни на что не повлияло. Значит, $y$ — фиктивная, а $f(x, y) = x$.
	\end{enumerate}

	Пример с формулой:
	$f(x, y) = (x \wedge y) \vee (x \wedge \bar{y})$
	Выносим $x$ за скобки: $x \wedge (y \vee \bar{y})$.
	Так как $(y \vee \bar{y}) = 1$, то $f = x \wedge 1 = x$.
	Переменная $y$ исчезла — она фиктивная.

	\chapter{Теорема о разложении функции по переменным (разложение по одной и по всем переменным).(35)}

	\begin{thbox}
		Любую функцию $f(x_1, \dots, x_i, \dots, x_n)$ можно представить в 
		виде:$$f(x_1, \dots, x_n) = (\bar{x}_i \wedge f(x_1, \dots, 0, \dots, x_n)) \vee (x_i \wedge 
		f(x_1, \dots, 1, \dots, x_n))$$
	\end{thbox}

	\begin{prbox}
		Доказательство теоремы о разложении по одной переменной  \\
		Тезис: Нужно доказать, что для любой булевой функции справедливо равенство: 
		$$f(x_1, \dots, x_i, \dots, x_n) = \bar{x}_i \cdot f(x_1, \dots, 0, \dots, x_n) \vee 
		x_i \cdot f(x_1, \dots, 1, \dots, x_n)$$Доказательство:Пусть $(a_1, a_2, \dots, a_n)$ — произвольный набор значений переменных. Рассмотрим два возможных случая для значения переменной $a_i$:Случай 1: $a_i = 0$Подставим это значение в правую часть равенства:$$\bar{0} \cdot f(a_1, \dots, 0, \dots, a_n) \vee 0 \cdot f(a_1, \dots, 1, \dots, a_n)$$Поскольку $\bar{0} = 1$, а $0 \cdot (\text{любое значение}) = 0$, получаем:$$1 \cdot f(a_1, \dots, 0, \dots, a_n) \vee 0 = f(a_1, \dots, 0, \dots, a_n)$$Это совпадает со значением левой части функции на данном наборе.Случай 2: $a_i = 1$Подставим это значение в правую часть равенства:$$\bar{1} \cdot f(a_1, \dots, 0, \dots, a_n) \vee 1 \cdot f(a_1, \dots, 1, \dots, a_n)$$Поскольку $\bar{1} = 0$, получаем:$$0 \vee 1 \cdot f(a_1, \dots, 1, \dots, a_n) = f(a_1, \dots, 1, \dots, a_n)$$Это также совпадает со значением левой части. 
		Вывод: Поскольку равенство справедливо для любого набора значений переменных, теорема доказана.
	\end{prbox}

	Если мы продолжим раскладывать функцию по каждой переменной одну за другой, 
	мы придем к Совершенной Дизъюнктивной Нормальной Форме (СДНФ).

	\chapter{ДНФ и КНФ. СДНФ и СКНФ. Правило получения СДНФ и СКНФ из вектор-столбца.(36)}
	\section{Определения ДНФ и КНФ}
	\begin{defbox}
		\begin{itemize}
			\item ДНФ (Дизъюнктивная нормальная 
			форма) — это дизъюнкция («ИЛИ») элементарных конъюнкций («И»). 
			Пример: $(x \wedge \bar{y}) \vee (y \wedge z)$.
			\item КНФ (Конъюнктивная нормальная форма) — это конъюнкция 
			(«И») элементарных дизъюнкций («ИЛИ»). 
			Пример: $(x \vee \bar{y}) \wedge (y \vee z)$.
		\end{itemize}
	\end{defbox}

	\section{СДНФ и СКНФ (Совершенные формы)}
	\begin{defbox}
		Форма называется Совершенной, если каждая элементарная 
		конъюнкция (или дизъюнкция) 
		содержит в себе все переменные, от которых зависит функция.
		
		\begin{itemize}
			\item СДНФ: Каждое слагаемое содержит все переменные. Каждому набору, 
			где функция равна $1$, соответствует ровно одна конъюнкция.
			\item СКНФ: Каждый множитель содержит все переменные. Каждому набору, 
			где функция равна $0$, соответствует ровно одна дизъюнкция.
		\end{itemize}
	\end{defbox}

	\section{Правило получения из вектор-столбца значений}
	\begin{defbox}
		Вектор-столбец — это просто значения функции из таблицы истинности, 
		записанные в столбик снизу вверх или сверху вниз.
	\end{defbox}
	Алгоритм получения СДНФ:\\
	\begin{itemize}
		\item Выделяем в вектор-столбце все единицы.
		\item Для каждой единицы смотрим на соответствующий 
		ей набор значений переменных $(x_1, x_2, \dots, x_n)$.
		\item Записываем конъюнкцию: если переменная в наборе 
		равна $1$, пишем её без отрицания, если $0$ — с отрицанием.
		\item Соединяем все полученные конъюнкции знаком дизъюнкции ($\vee$).
	\end{itemize}

	Алгоритм получения СКНФ: \\

	\begin{itemize}
		\item Выделяем в вектор-столбце все нули.
		\item Для каждого нуля смотрим на набор значений переменных.
		\item Записываем дизъюнкцию: если переменная в наборе равна $0$, 
		пишем её без отрицания, если $1$ — 
		с отрицанием (это интуитивно «наоборот» по сравнению с СДНФ).
		\item Соединяем все полученные дизъюнкции знаком конъюнкции ($\wedge$).
	\end{itemize}

	\chapter{Булевы операции. Булева алгебра. Основные свойства булевых операций.(36)}
	\section{Определение Булевой алгебры}
	\begin{defbox}
		Булева алгебра — это алгебраическая структура 
		$\langle B, \vee, \wedge, \neg, 0, 1 \rangle$, 
		состоящая из множества $B$ (элементы которого 
		называются логическими значениями), двух бинарных 
		операций — дизъюнкции ($\vee$) и конъюнкции ($\wedge$), 
		одной унарной операции — отрицания ($\neg$), и двух выделенных 
		констант: 0 (логический ноль) и 1 (логическая единица).\\В контексте 
		двузначной логики множество $B = \{0, 1\}$. 
		Операции определяются следующими правилами:
		\begin{itemize}
			\item Конъюнкция ($x \wedge y$): принимает значение 1 
			тогда и только тогда, когда оба аргумента равны 1.
			\item Дизъюнкция ($x \vee y$): принимает значение 1, 
			если хотя бы один из аргументов равен 1.
			\item Отрицание ($\neg x$): меняет значение 
			аргумента на противоположное ($\neg 0 = 1, \neg 1 = 0$).
		\end{itemize}
	\end{defbox}

	\section{Аксиоматика и свойства булевых операций}
	Для любых элементов $x, y, z \in B$ справедливы следующие тождества:

	\subsection*{Группа 1: Коммутативность, ассоциативность и дистрибутивность}

	\textbf{Коммутативность:}
	\begin{align*}
	x \vee y &= y \vee x \\
	x \wedge y &= y \wedge x
	\end{align*}

	\textbf{Ассоциативность:}
	\begin{align*}
	(x \vee y) \vee z &= x \vee (y \vee z) \\
	(x \wedge y) \wedge z &= x \wedge (y \wedge z)
	\end{align*}

	\textbf{Дистрибутивность (распределительный закон):}
	\begin{align*}
	x \wedge (y \vee z) &= (x \wedge y) \vee (x \wedge z) \quad \text{(конъюнкции относительно дизъюнкции)} \\
	x \vee (y \wedge z) &= (x \vee y) \wedge (x \vee z) \quad \text{(дизъюнкции относительно конъюнкции)}
	\end{align*}

	\subsection*{Группа 2: Законы идемпотентности и поглощения}

	\textbf{Идемпотентность:}
	\begin{align*}
	x \vee x &= x \\
	x \wedge x &= x
	\end{align*}

	\textbf{Поглощение:}
	\begin{align*}
	x \vee (x \wedge y) &= x \\
	x \wedge (x \vee y) &= x
	\end{align*}

	\subsection*{Группа 3: Свойства констант и инверсии}

	\textbf{Операции с константами ($0$ и $1$):}
	\begin{align*}
	x \vee 0 &= x, & x \vee 1 &= 1 \\
	x \wedge 1 &= x, & x \wedge 0 &= 0
	\end{align*}

	\textbf{Законы исключенного третьего и противоречия:}
	\begin{align*}
	x \vee \neg x &= 1 \\
	x \wedge \neg x &= 0
	\end{align*}

	\textbf{Закон двойного отрицания:}
	\[
	\neg(\neg x) = x
	\]

	\subsection*{Группа 4: Законы де Моргана}

	\textbf{Отрицание сложных выражений:}
	\begin{align*}
	\neg(x \wedge y) &= \neg x \vee \neg y \\
	\neg(x \vee y) &= \neg x \wedge \neg y
	\end{align*}

	\section{Принцип двойственности}
	Для любого верного логического тождества справедливо 
	двойственное ему тождество, полученное путем взаимной замены 
	операций дизъюнкции ($\vee$) на конъюнкцию ($\wedge$) и констант 
	$0$ на $1$ (и наоборот). Это свойство симметрии 
	подчеркивает равноправие операций в структуре булевой алгебры.

	\chapter{Законы: де Моргана, поглощения, склеивания, расщепления.(38)}
	\section{Законы де Моргана}
	Законы де Моргана устанавливают связь между отрицанием, 
	конъюнкцией и дизъюнкцией. Они позволяют переходить от 
	отрицания всей логической операции к отрицанию отдельных переменных.

	\begin{itemize}
		\item Для конъюнкции: Отрицание конъюнкции 
		равно дизъюнкции отрицаний.
		$$\overline{x \cdot y} = \bar{x} \vee \bar{y}$$
		\item Для дизъюнкции: Отрицание 
		дизъюнкции равно конъюнкции отрицаний.
		$$\overline{x \vee y} = \bar{x} \cdot \bar{y}$$
	\end{itemize}

	\section{Законы поглащения}
	\begin{defbox}
		Законы поглощения позволяют упрощать 
		выражения, в которых 
		одна переменная (или подвыражение) 
		входит как в качестве отдельного 
		операнда, так и в состав другого операнда.
	\end{defbox}
	\begin{itemize}
		\item Первый закон поглощения:
		$$x \vee (x \cdot y) = x$$
		\item Второй закон поглощения:
		$$x \cdot (x \vee y) = x$$
	\end{itemize}

	\section{Законы склеивания}
	\begin{defbox}
		Законы склеивания являются фундаментальными 
		для минимизации булевых функций. 
		Они позволяют исключать переменную, 
		если она входит в две конъюнкции 
		(или дизъюнкции) в прямом и инверсном 
		виде при неизменности остальных частей.
	\end{defbox}

	\begin{itemize}
		\item Для СДНФ (склеивание по конъюнкциям):
		$$(x \cdot K) \vee (\bar{x} \cdot K) = K$$
		(Где $K$ — любая элементарная конъюнкция)
		\item Для СКНФ (склеивание по дизъюнкциям):
		$$(x \vee D) \cdot (\bar{x} \vee D) = D$$
		(Где $D$ — любая элементарная дизъюнкция)
	\end{itemize}

	\section{Законы расщепления (Обобщенная дистрибутивность)}
	\begin{itemize}
		\item Расщепление по переменной (прямое):
		$$x = (x \cdot y) \vee (x \cdot \bar{y})$$
		\item Расщепление (двойственное):
		$$x = (x \vee y) \cdot (x \vee \bar{y})$$
	\end{itemize}

	\chapter{Имплицирование функции. 
	Импликант, простой импликант, сокращенная ДНФ. 
	Проверка импликант на простоту.(39)}

	\section{Понятие импликанта}
	\begin{defbox}
		Элементарная конъюнкция $K$ называется 
		импликантом функции $f$, если из истинности 
		$K$ следует истинность $f$.
		Математически это записывается как: 
		$K \to f \equiv 1$ (или $K \le f$).
	\end{defbox}

	\section{Простой импликант}
	\begin{defbox}
		Импликант $K$ функции $f$ называется простым, если 
		после удаления из него любой переменной (любого литерала) 
		полученная 
		конъюнкция перестает быть импликантом этой функции.
	\end{defbox}
	\section{Сокращенная ДНФ}
	\begin{defbox}
		Сокращенная ДНФ — это дизъюнкция 
		всех простых импликантов данной функции.
	\end{defbox}
	\section{Проверка импликанта на простоту}
	Для проверки того, является ли импликант 
	$K$ простым, используется метод 
	вычеркивания переменных:
	\begin{enumerate}
		\item Пусть $K = x_1^{\sigma_1} \cdot x_2^{\sigma_2} \dots x_m^{\sigma_m}$.
		\item Поочередно удаляем по одной переменной из конъюнкции $K$.
		\item Для каждой полученной укороченной конъюнкции $K'$ проверяем условие $K' \le f$.
		\item Результат:
		\begin{itemize}
			\item Если хотя бы для одной $K'$ условие 
			$K' \le f$ выполняется, то исходный 
			импликант $K$ не является 
			простым (его можно сократить).
			\item Если ни одна сокращенная конъюнкция 
			не является импликантом функции $f$, 
			то $K$ — простой импликант.
		\end{itemize}
	\end{enumerate}

	\chapter{Получение сокращенной ДНФ методом Блейка-Порецкого. (40)}
	\section{Теоретическая основа метода}
	Метод базируется на использовании двух 
	основных операций: обобщенного 
	склеивания и поглощения.
	\begin{itemize}
		\item Операция обобщенного склеивания:Если функция 
		представлена в виде $f = Ax \vee B\bar{x} \vee 
		\Phi$, то к ней можно добавить конъюнкцию $(A \cdot B)$, 
		называемую консенсусом (или логическим следствием) 
		двух исходных конъюнкций.
		$$Ax \vee B\bar{x} = Ax \vee B\bar{x} \vee AB$$
		\item Операция поглощения:Если 
		в выражении присутствуют 
		конъюнкции $K_1$ и $K_2$ такие, 
		что $K_1 \subseteq K_2$ (все 
		литералы $K_1$ входят в $K_2$), 
		то $K_2$ удаляется:
		$$K_1 \vee K_1 K_2 = K_1$$
	\end{itemize}
	\section{Алгоритм Блейка-Порецкого}
	Процесс получения сокращенной ДНФ состоит из двух этапов:\\

	
	Этап I: Применение правила обобщенного склеивания
	К исходной ДНФ последовательно применяются
 	все возможные операции обобщенного 
	склеивания до тех пор, пока это возможно.

	Этап II: Применение правила поглощения
	После того как новые 
	консенсусы перестают порождаться, 
	из полученной ДНФ удаляются все 
	поглощаемые конъюнкции.

	\chapter{ Двойственная функция. Самодовойственная функция. Принцип двойственности.(41)}
	\section{Двойственная функция}
	\begin{defbox}
		Функция $f^*(x_1, x_2, \dots, x_n)$ называется двойственной
		 к функции $f(x_1, x_2, \dots, x_n)$, если она 
		 получается путем инвертирования всех аргументов и самого значения 
		функции:$$f^*(x_1, x_2, \dots, x_n) = \neg f(\neg x_1, \neg x_2, \dots, \neg x_n)$$
	\end{defbox}
	\textbf{Свойства двойственности:}
	\begin{itemize}
		\item Если заменить в выражении функции все операции $\vee$ на $\wedge$, $\wedge$ на $\vee$, а константы $0$ на $1$ и $1$ на $0$, получится двойственная функция.
		\item Пример: Для функции конъюнкции $f = x \wedge y$ двойственной будет дизъюнкция $f^* = x \vee y$.
		\item Справедливо соотношение: $(f^*)^* = f$ (принцип взаимности).
	\end{itemize}

	\section{Самодвойственная функция}
	\begin{defbox}
		Функция называется самодвойственной, если она равна своей двойственной функции:
		$$f = f^*$$То есть: $f(x_1, \dots, x_n) = \neg f(\neg x_1, \dots, \neg x_n)$.
	\end{defbox}
	\section{Принцип двойственности}
	Этот принцип утверждает: если 
	в логическом тождестве заменить 
	каждую функцию на двойственную 
	ей, то полученное равенство также будет верным. \\
	Это означает:\\
	\begin{enumerate}
		\item Замените все $\vee$ на $\wedge$.
		\item Замените все $\wedge$ на $\vee$.
		\item Замените все $0$ на $1$, а $1$ на $0$.
	\end{enumerate}

	\chapter{Получение двойственной 
	функции по определении. Приведение ее к ДНФ. 
	Определение самодовойственности.(42)}
	\section{Получение двойственной функции по определению}
	тобы получить $f^*$, нужно взять отрицание от функции, где все аргументы также инвертированы:
	$$f^*(x_1, \dots, x_n) = \neg f(\neg x_1, \dots, \neg x_n)$$
	\section{Приведение к ДНФ}
	Для этого используем законы алгебры логики, чтобы выражение 
	имело вил конечной дизъюнктивной формы, где каждая из элементырных 
	конъюнкций входит не более одного раза, а все элементарные 
	конъюнкции связаны дизъюнкциями.
	\section{Определение самодвойственности}
	\begin{defbox}
		Функция называется самодвойственной, если $f = f^*$. Проверить это можно двумя способами:\\
		\begin{enumerate}
			\item Сравнение выражений
			\item Таблица истинности
		\end{enumerate}
	\end{defbox}
	\chapter{Алгебра Жегалкина. Эквивалентные формулы 
	алгебры Жегалкина. Формулы 
	перехода от булевой формулы к полиному Жегалкина.(43)}
	\section{Основы алгебры Жегалкина}
	\begin{defbox}
		В этой системе базис состоит из набора функций $\{ \wedge, \oplus, 1 \}$, где:\\
		\begin{itemize}
			\item Конъюнкция (умножение): $x \wedge y$
			\item Сложение по модулю 2 (XOR): $x \oplus y$
		\end{itemize}
	\end{defbox}
	Свойства операций:\\
	\begin{enumerate}
		\item Коммутативность и ассоциативность для обеих операций.
		\item Дистрибутивность: $x(y \oplus z) = xy \oplus xz$.
		\item Идемпотентность умножения: $x \cdot x = x$.
		\item Свойство нуля: $x \oplus x = 0$ и $x \oplus 0 = x$.
	\end{enumerate}
	\section{Полином Жегалкина}
	Любую булеву функцию можно представить в виде суммы 
	(по модулю 2) различных конъюнкций переменных. 
	Это представление единственно (с точностью до порядка слагаемых).
	$$P = a_0 \oplus a_1 x_1 \oplus a_2 x_2 \oplus \dots \oplus 
	a_{12} x_1 x_2 \oplus \dots \oplus a_{1\dots n} x_1 \dots x_n$$Где 
	коэффициенты $a_i$ принимают значения $0$ или $1$.
	\section{Формулы перехода от булевой формулы к полиному}
	Правила замены: \\ 
	\begin{itemize}
		\item Отрицание: $\neg x = x \oplus 1$
		\item Конъюнкция: $x \wedge y = xy$
		\item Дизъюнкция: $x \vee y = x \oplus y \oplus xy$
		\item Импликация: $x \to y = 1 \oplus x \oplus xy$
		\item Эквивалентность: $x \leftrightarrow y = 1 \oplus x \oplus y$
	\end{itemize}

	\section{Методы построения полинома}
	\begin{enumerate}
		\item Метод равносильных преобразований
		\item Метод неопределенных коэффициентов
		Вы записываете общий вид полинома и подставляете в него значения функции из таблицы истинности, решая систему уравнений.
		(То что разбирали на лекции и практике)
		\item Метод преобразования треугольником (Метод Паскаля)
		Это графический способ, где коэффициенты 
		полинома находятся путем последовательного сложения значений таблицы истинности.
	\end{enumerate}

	\chapter{Получение полинома Жегалкина. Определение линейности функции.(44)}
	\section{Получение полинома Жегалкина}
	\begin{enumerate}
		\item Аналитические подстановки
		Используются следующие формулы:
		\begin{itemize}
			\item $x \wedge y \to xy$
			\item $\neg x \to x \oplus 1$
			\item $x \vee y \to x \oplus y \oplus xy$
			\item $x \to y \to x \oplus 1 \oplus xy$
		\end{itemize}
		\item Составление общего выида через таблицу истинности и решение уравнений
		\item Метод треугольника
		\begin{enumerate}
			\item Выписываете столбец значений функции $f$.
			\item Рядом рисуете новый столбец, где каждый элемент — это сумма по модулю 2 двух соседних элементов из предыдущего столбца.
			\item Повторяете, пока не останется один элемент.
			\item Коэффициенты полинома — это самые верхние числа в каждом столбце.
		\end{enumerate}
	\end{enumerate}
	\section{Определение линейности функции}
	\begin{defbox}
		Функция называется линейной, если в её полиноме 
		Жегалкина отсутствуют произведения переменных. 
		То есть переменные встречаются только в первой степени и соединяются операцией $\oplus$.Общий вид линейной функции:
		$$f(x_1, x_2, \dots, x_n) = a_0 \oplus a_1x_1 \oplus a_2x_2 \oplus \dots \oplus a_nx_n$$где $a_i \in \{0, 1\}$.
	\end{defbox}
	Как проверить функцию на линейность?
	\begin{enumerate}
		\item Построить полином Жегалкина. Если в нем 
		есть слагаемые типа $xy$, $xyz$ или 
		любые другие произведения двух и более переменных — функция нелинейная.
		\item По таблице истинности. Для линейных функций (кроме констант) 
		количество наборов, на которых функция равна 1, всегда равно 
		$2^{n-1}$ (ровно половина таблицы). 
		Однако это необходимое, но не достаточное условие.
	\end{enumerate}

	\chapter{Теорема Шеннона}
	\begin{thbox}
		{[Теорема Шеннона]}

		Любую логическую функцию можно разложить по одной или нескольким переменным:
		$$
		f(x_1,\dots,x_n) = x_if_{x_i=1} \oplus \lnot x_if_{x_i=0}
		$$
	\end{thbox}
	\begin{prbox}
		Проверим разложение. Подставим $x_i = 1$, $x_i = 0$:
		$$
		x_if_{x_i=1} \oplus \lnot x_if_{x_i=0} = 1 \cdot f_{x_i=1} \oplus 0 \cdot f_{x_i=0} = f_{x_i=1}
		$$
		$$
		x_if_{x_i=1} \oplus \lnot x_if_{x_i=0} = 0 \cdot f_{x_i=1} \oplus 1 \cdot f_{x_i=0} = f_{x_i=0}
		$$
	\end{prbox}
	\chapter{Теорема о представлении булевой функции в СДНФ и СКНФ}
	\begin{thbox}
		Любая булева функция, не равная тождественно нулю, может быть однозначно представлена в виде совершенной дизъюнктивной нормальной формы (СДНФ), а любая функция, не равная тождественно единице, — в виде совершенной конъюнктивной нормальной формы (СКНФ).
	\end{thbox}
	\begin{prbox}
		Рассмотрим таблицу истинности функции $f$.

		\textbf{Для СДНФ:}

		Выберем все стоки, где $f=1$. Для каждой такой строки построим конъюнкцию всех переменных: если переменная равна $1$, то берем без отрицания. Иначе -- с ним.

		\textbf{Для СКНФ:}

		Выберем все строки, где $f = 0$. Для каждой такой строки построим дихъюнкцию всех переменных: если переменная равна $0$, то берем без отрицания. Иначе -- с ним.

		Единственность следует из того, что таблица истинности однозначно задает функцию
	\end{prbox}
	\chapter{Принцип двойственности в булевой алгебре}
	\begin{thbox}
		Если в верном тождестве булевой алгебры заменить все операции на двойственные, $0$ на $1$ и $1$ на $0$, то получится также верное тождество.
	\end{thbox}
	\begin{prbox}
		Если $f=g$, то их двойственные также равны: $f*=g*$. Но при замене из формулировки теоремы от функций $f$ и $g$ происходит переход к их двойственным, которые также равны.
	\end{prbox}

\end{document}