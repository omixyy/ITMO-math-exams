\documentclass[14pt]{extreport}
\usepackage{graphicx}
\usepackage{gost}
\usepackage[T1,T2A]{fontenc}
\usepackage[utf8]{inputenc}
\PassOptionsToPackage{russian,english}{babel}
\usepackage[english,russian]{babel}
\usepackage{tempora}
\usepackage{hyperref}
\linespread{1.3}
\setlength{\parindent}{1.25cm}
\usepackage{hyperref}

\usepackage{multirow}
\usepackage{makecell}
\usepackage{longtable}
\usepackage{array}
\newcolumntype{P}[1]{>{\raggedright\arraybackslash}p{#1}}

\usepackage{fancyhdr}
\pagestyle{fancy}
\fancyhf{} 
\fancyhead{} 
\fancyfoot[C]{\thepage} 

\usepackage{amsmath, amssymb}
\usepackage{tcolorbox}
\usepackage{enumitem}
\usepackage{geometry}
\usepackage{amsthm}

\newtcolorbox{thbox}[1][]{colback=orange!10, colframe=orange!60, title=Теорема, #1}
\newtcolorbox{prbox}[1][]{colback=green!5, colframe=green!60!gray, title=Доказательство, #1}
\newtcolorbox{defbox}[1][]{colback=blue!5, colframe=blue!60, title=Определение, #1}

\newcommand{\defeq}[0]{\stackrel{\text{def}}{=}}

\begin{document}
\thispagestyle{empty}
	\title{Математический анализ. Экзамен.}
	\maketitle

	\newpage
	\tableofcontents
	
	\chapter{Способы задания множества. Порождающая процедура. Характеристическое свойство.}

	\section{Способы задания множества}

	\begin{enumerate}
		\item Перечисление $\lbrace a, b, c ... \rbrace$
		\item Характеристическое свойство $M \stackrel{\text{def}}{=} \lbrace x : P(x) \rbrace$, где $P(x)$ -- предикат
		\item Порождающая процедура $M := \lbrace y : y = f(x), x \in E \rbrace$, где $f$ -- функция от $x$
	\end{enumerate}
	\section{Описание способов задания множества}

	\begin{defbox}
		\textbf{Характеристическое свойство} -- способ задания множества, при котором каждый его элемент обладает свойством $P(x)$

		$$
		M \defeq \lbrace x : P(x) \rbrace
		$$
	\end{defbox}

	\begin{defbox}
		\textbf{Порождающая процедура} -- способ задания множества, при котором каждый его элемент является результатом выполнения функции $f$ от переменной $x$ из некоторого множества $E$.

		$$
		M := \lbrace y : y = f(x), x \in E \rbrace
		$$
	\end{defbox}

	\chapter{Отображения. Инъекция, сюръекция, биекция. Прямые произведения множеств}

	\section{Отображения}

	\begin{defbox}
		Отображение -- способ сопоставления элементов между множествами.
	\end{defbox}

	\noindent Отображения бывают трех видов:
	\begin{enumerate}
		\item Инъекция
		\item Сюръекция
		\item Биекция
	\end{enumerate}

	\begin{defbox}
		\textbf{Инъекция} -- такое отображение $f(A) \to B$, при котором любой элемент $B$ имеет \textbf{не более одного} прообраза в множестве $A$.
	\end{defbox}

	\begin{defbox}
		\textbf{Сюръекция} -- такое отображение $f(A) \to B$, при котором любой элемент $B$ имеет \textbf{не менее одного} прообраза в множестве $A$.
	\end{defbox}

	\begin{defbox}
		\textbf{Биекция} -- это отображение, являющееся и сюръекцией, и инъекцией одновременно. То есть взаимооднозначное соответствие.
	\end{defbox}

	\section{Прямые произведения множеств}

	\begin{defbox}
		\textbf{Прямое (декартово) произведение множеств} $A$ и $B$ -- все такие пары чисел $(a, b)$, где $a \in A$, $b \in B$.

		$$
		A \times B \defeq \lbrace (a, b) : a \in A, b \in B \rbrace
		$$
	\end{defbox}
\end{document}