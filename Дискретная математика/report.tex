\documentclass[14pt]{extreport}
\usepackage{graphicx}
\usepackage{gost}
\usepackage[T1,T2A]{fontenc}
\usepackage[utf8]{inputenc}
\PassOptionsToPackage{russian,english}{babel}
\usepackage[english,russian]{babel}
\usepackage{tempora}
\usepackage{hyperref}
\linespread{1.3}
\setlength{\parindent}{1.25cm}
\usepackage{hyperref}

\usepackage{multirow}
\usepackage{makecell}
\usepackage{longtable}
\usepackage{array}
\newcolumntype{P}[1]{>{\raggedright\arraybackslash}p{#1}}

\usepackage{fancyhdr}
\pagestyle{fancy}
\fancyhf{} 
\fancyhead{} 
\fancyfoot[C]{\thepage} 

\usepackage{amsmath, amssymb}
\usepackage{tcolorbox}
\usepackage{enumitem}
\usepackage{geometry}
\usepackage{amsthm}

\tcbuselibrary{breakable}

\newtcolorbox{thbox}[1][]{colback=orange!10, colframe=orange!60, title=Теорема, #1, breakable=true, toprule at break=0pt, % верхняя рамка не показывается при разрыве
    bottomrule at break=0pt}
\newtcolorbox{prbox}[1][]{colback=green!5, colframe=green!60!gray, title=Доказательство, breakable=true, #1, toprule at break=0pt, % верхняя рамка не показывается при разрыве
    bottomrule at break=0pt}
\newtcolorbox{defbox}[1][]{colback=blue!5, colframe=blue!60, title=Определение, #1, breakable=true, toprule at break=0pt, % верхняя рамка не показывается при разрыве
    bottomrule at break=0pt}
\newtcolorbox{impbox}[1][]{colback=red!5, colframe=red!60, title=Важно, #1, breakable=true, toprule at break=0pt, % верхняя рамка не показывается при разрыве
    bottomrule at break=0pt}

\newtcolorbox{rulebox}[1][]{colback=purple!5, colframe=purple!60, title=Правило, #1, breakable=true, toprule at break=0pt, % верхняя рамка не показывается при разрыве
    bottomrule at break=0pt}

\newcommand{\defeq}[0]{\stackrel{\text{def}}{=}}

\begin{document}
\thispagestyle{empty}
	\title{Математический анализ. Экзамен.}
	\maketitle

	\newpage
	\tableofcontents
	
	\chapter{Способы задания множества. Порождающая процедура. Характеристическое свойство.}

	\section{Способы задания множества}

	\begin{enumerate}
		\item Перечисление $\lbrace a, b, c ... \rbrace$
		\item Характеристическое свойство $M \stackrel{\text{def}}{=} \lbrace x : P(x) \rbrace$, где $P(x)$ -- предикат
		\item Порождающая процедура $M := \lbrace y : y = f(x), x \in E \rbrace$, где $f$ -- функция от $x$
	\end{enumerate}
	\section{Описание способов задания множества}

	\begin{defbox}
		\textbf{Характеристическое свойство} -- способ задания множества, при котором каждый его элемент обладает свойством $P(x)$

		$$
		M \defeq \lbrace x : P(x) \rbrace
		$$
	\end{defbox}

	\begin{defbox}
		\textbf{Порождающая процедура} -- способ задания множества, при котором каждый его элемент является результатом выполнения функции $f$ от переменной $x$ из некоторого множества $E$.

		$$
		M := \lbrace y : y = f(x), x \in E \rbrace
		$$
	\end{defbox}

	\chapter{Отображения. Инъекция, сюръекция, биекция. Прямые произведения множеств}

	\section{Отображения}

	\begin{defbox}
		Отображение -- способ сопоставления элементов между множествами.
	\end{defbox}

	\noindent Отображения бывают трех видов:
	\begin{enumerate}
		\item Инъекция
		\item Сюръекция
		\item Биекция
	\end{enumerate}

	\begin{defbox}
		\textbf{Инъекция} -- такое отображение $f(A) \to B$, при котором любой элемент $B$ имеет \textbf{не более одного} прообраза в множестве $A$.
	\end{defbox}

	\begin{defbox}
		\textbf{Сюръекция} -- такое отображение $f(A) \to B$, при котором любой элемент $B$ имеет \textbf{не менее одного} прообраза в множестве $A$.
	\end{defbox}

	\begin{defbox}
		\textbf{Биекция} -- это отображение, являющееся и сюръекцией, и инъекцией одновременно. То есть взаимооднозначное соответствие.
	\end{defbox}

	\section{Прямые произведения множеств}

	\begin{defbox}
		\textbf{Прямое (декартово) произведение множеств} $A$ и $B$ -- все такие пары чисел $(a, b)$, где $a \in A$, $b \in B$.

		$$
		A \times B \defeq \lbrace (a, b) : a \in A, b \in B \rbrace
		$$
	\end{defbox}

	\chapter{Мощность множества}
	\begin{defbox}
		\textbf{Мощность множества} -- количество элементов в нем (для конечных множеств). 
		
		Для бесконечных:
		\begin{enumerate}
			\item Если между бесконечным множеством $X$ и множеством натуральных чисел $\mathbb{N}$ существует биекция, то говорят, что $X$ имеет счётную мощность. Это "наименьшая" бесконечная мощность.
			\item Если между $X$ и множеством всех вещественных чисел $\mathbb{R}$ (или отрезком $[0,1]$) существует биекция, то говорят, что $X$ имеет мощность континуума.
		\end{enumerate}
	\end{defbox}

	\chapter{Операции над множествами.}

	\section{Операции}

	\begin{enumerate}
		\item $A\cap B$ -- пересечение 
		$$
		A \cap B \defeq \lbrace x : (x\in A) \wedge (x\in B) \rbrace
		$$
		\item $A\cup B$ -- объединение
		$$
		A\cap B \defeq \lbrace x : (x\in A) \vee (x \in B) \rbrace
		$$
		\item $A \setminus B$ -- разность
		$$
		A \setminus B \defeq \lbrace x : (x \in A) \wedge (x \notin B)\rbrace
		$$
		\item $A\triangle B$ -- симметрическая разность
		$$
		A \setminus B \defeq \lbrace x : (x \in (A \setminus B)) \vee (x \in (B \setminus A))\rbrace
		$$
		\item $\bar A$ -- дополнение до универсума
		$$
		\bar A \defeq \lbrace x : (x \notin A) \wedge (x \in U) \rbrace
		$$

	\end{enumerate}

	\section{Свойства}
	\begin{enumerate}
		\item $A \cap A = A$, $A \cup A = A$
		\item $A \cap (B \cap C) = (A \cap B) \cap C$,
				
				$A \cup (B \cup C) = (A \cup B) \cup C$
		\item $A \cap (B \cup C) = (A \cap B) \cup (A \cap C)$, 
		
				$A \cup (B \cap C) = (A \cup B) \cap (A \cup C)$ 

		\item $A \cup \emptyset = A$, $A \cap \emptyset = \emptyset$
		\item $A \cup U = U$, $A \cap U = A$
		\item $A \cap B = B \cap A$, $A \cup B = B \cup A$
		\item $A \cup B = A \cap \bar{B}$,  
		
				$\overline{A \cap B} = \bar{A} \cup \bar B$
		\item $\bar{ \bar A} = A$
		\item $A \setminus B = A \cap \bar B$
	\end{enumerate}

	\chapter{Прямое произведение $N$ множеств.}

	\begin{defbox}
		\textbf{Прямое произведение $n$ множеств} -- все возможные кортежи из элементов этих $n$ множеств.

		На примере двух множеств:
		$$
		A \times B = \lbrace (a, b) : a\in A, b \in B \rbrace
		$$
	\end{defbox}

	\chapter{Теорема о мощности прямого произведения множеств}

	\begin{thbox}
		Если $A$ и $B$ конечны, $|A| = n$, $|B| = m$ то $|A \times B| = |A| |B| = mn$
	\end{thbox}

	\begin{prbox}
		Рассмотрим кортеж. В нем на первом месте стоит элемент из $A$, на втором -- из $B$. К каждому элементу из $A$ можно приставить $m$ элементов из $B$, получив тем самым множество, представляющее собой результат декартового произведения. То есть на первом месте в кортеже может стоять $n$ элементов, на втором -- $m$. Значит всего таких кортежей можно составить $mn$ штук.
	\end{prbox}

	\chapter{Понятие вектора. Проекция вектора на оси. Проекция множества векторов}

	\begin{defbox}
		Пусть задано прямое произведение $A_1 \times A_2 \times ... \times A_n$. \textbf{Вектором} называется упорядоченный набор $(a_1, a_2, ... a_n)$, где $a_i \in A_i$
	\end{defbox}

	\begin{defbox}
		Пусть задано прямое произведение $A_1 \times A_2 \times ... \times A_n$. \textbf{Проецированием $\operatorname{Pr}_k(a_1, a_2, ... a_k...a_n)$} называется отображение $(a_1, a_2, ... a_n) \rightarrow (a_k)$
		$$
		\operatorname{Pr}_k(a_1, a_2, ... a_k...a_n) \defeq f((a_1, a_2, ... a_k...a_n)) \rightarrow (a_k)
		$$
		Проецирование множества векторов:
		$$
		\operatorname{Pr}_{i, j, ... m}(\lbrace (a_1, a_2,...a_m), (b_1, b_2,...b_m)... \rbrace) = 
		$$
		$$
		= \lbrace (a_i, a_j,...a_m), (b_i, b_j, ... b_m)... \rbrace
		$$
	\end{defbox}

	\chapter{Правило суммы. Правило произведения.}

	\begin{thbox}
		{[Правило суммы]}

		Пусть все выборки из множества $A$ делятся на две взаимоисключающие $A_1$ и $A_2$. Число выборок первого типа $m_1$, второго -- $m_2$. Тогда число всех выборок из множества $A$ равно $m_1 + m_2$
	\end{thbox}

	\begin{prbox}
		Данное правило является следствием формулы включений-исключений:
		$$
		|A \cup B| = |A| + |B| - |A \cap B|
		$$
		где $A \cap B = \emptyset$
	\end{prbox}

	\begin{thbox}
		{[Правило произведения]}
		
		Пусть число способов построить выборку из множества $A$ равно $n$, из множества $B$ -- $m$. Тогда число способов построить выборку $(a, b)$ ($a \in A, b \in B$) равно $mn$
	\end{thbox}

	\begin{prbox}
		Данное утверждение эквивалентно теореме о мощности прямого произведения. $|A \times B| = |A| |B|=mn$
	\end{prbox}

	\chapter{Число размещений без повторений, число размещений с повторениями}

	\begin{defbox}
		Пусть имеется множество из $n$ элементов. Упорядоченное подмножество из $k$ элементов называется размещением без повторений
		
	\end{defbox}
	\begin{thbox}
		Число размещений без повторений можно рассчитать по формуле:
		$$
		A_n^k = \frac{n!}{(n-k)!}
		$$
	\end{thbox}

	\begin{prbox}
		Пусть есть $n$ элементов, из которых нужно составить упорядоченные наборы из $k$ элементов.

		Тогда на первое место можно поставить $n$ элементов, на второе -- $n-1$, на третье -- $n-2$ и так далее до $k$-го места, куда можно поставить $n-k+1$ элементов. тогда посчитаем общее количество наборов по правилу произведения: 
		$$
		n \cdot (n-1) \cdot (n-2) ... (n-k+1) = \frac{n!}{(n-k)!}
		$$
	\end{prbox}

	\vspace{20pt}
	\begin{defbox}
		Пусть имеется множество $A$ из $n$ элементов. Набор $(m_1, m_2 ... m_k)$, где $\forall i \Rightarrow m_i \in A$, называется размещением с повторениями.
	\end{defbox}

	\begin{thbox}
		Число размещений с повторениями можно рассчитать по формуле:
		$$
		\overline{A_n^k} = n^k
		$$
	\end{thbox}

	\begin{prbox}
		Есть $k$ позиция, на каждой тз них $n$ элементов. Тогда по правилу произведения всего $n^k$ наборов
	\end{prbox}

	\chapter{Число перестановок без повторений, число перестановок с повторениями.}

	\begin{defbox}
		Пусть имеется множество из $n$ элементов. \textbf{Перестановкой с повторениями} называется упорядоченная последовательность его элементов
	\end{defbox}

	\begin{thbox}
		Перестановки без повторений можно посчитать по формуле:
		$$
		P_n = n!
		$$
	\end{thbox}
	\begin{prbox}
		Данная формула является следствием правила произведения. Есть $n$ позиций, на каждой следующей на $1$ элемент меньше, чем на предыдущей. Значит формула:
		$$
		P_k = n!
		$$
	\end{prbox}

	\vspace{20pt}
	\begin{defbox}
		Пусть имеется множество из $n$ элементов, среди которых:
		\begin{itemize}
			\item $k_1$ неразличимых элементов 1-го типа
			\item $k_2$ неразличимых элементов 2-го типа
			
			\dots

			\item $k_s$ неразличимых элементов s-го типа
		\end{itemize}
		\textbf{Перестановкой с повторениями} называется упорядоченная последовательность элементов этого множества 
	\end{defbox}

	\begin{thbox}
		Число перестановок с повторениями можно рассчитать по формуле:
		$$
		\overline{P_n} = \frac{(k_1 + k_2 + ... + k_s)!}{k_1!k_2!...k_s!}
		$$
		где $k_1+k_2+...+k_s=n$
	\end{thbox}
	\begin{prbox}
		Для начала сосчитаем количество перестановок без повторений (представим, что в исходном множестве разные элементы): $n!$. Теперь сосчитаем перестановки для каждой группы: $k_i!$. Поскольку в исходном множестве есть группы одинаковых элементов, то их перестановки для нас неразличимы, а значит необходимо убрать все перестановки, в которых одинаковые наборы элементов одного типа, а их, как мы сосчитали, для каждого типа $k_i!$. Итого формула:
		$$
		\overline{P_n} = \frac{n!}{k_1!k_2!...k_s!}
		$$
	\end{prbox}

	\chapter{Число сочетаний без повторений, число сочетаний с повторениями.}

	\section{Сочетания без повторений}
	\begin{defbox}
		Пусть имеется множество из $n$ элементов. Неупорядоченное подмножество из $k$ его элементов называется \textbf{сочетанием без повторений}
	\end{defbox}

	\begin{thbox}
		Число сочетаний без повторений рассчитывается по формуле:
		$$
		C_n^k = \frac{n!}{(n-k)!k!}
		$$
	\end{thbox}

	\begin{prbox}
		Посмотрим на формулу размещений без повторений: $A_n^k = \frac{n!}{(n-k)!}$. Чтобы рассчитать сочетания, небходимо убрать из размещений все те наборы, в которых совпадают элементы. Для каждого набора элементов таких повторяющихся наборов $k!$, ведь это просто перестановки. Тогда итоговая формула:
		$$
		C_n^k = \frac{n!}{(n-k)!k!}
		$$
	\end{prbox}

	\section{Сочетания с повторениями}
	\begin{defbox}
		Пусть имеется $k$ классов элементов множества $A$. Сочетанием с повторениями называется неупорядоченная выборка $n$ элементов из множества $A$.
	\end{defbox}

	\begin{thbox}
		Число сочетаний с повторениями можно рассчитать по формуле:
		$$
		\overline{C_n^k} = C_{n+k-1}^{k-1}=\frac{(n+k-1)!}{n!(k-1)!}
		$$
	\end{thbox}

	\begin{prbox}
		Используем метод "звезд и перегородок": будем рассматривать $k$ звезд и $n-1$ перегородок для $n$ типов звезд. Любая такая последовательность однозначно задаёт одно сочетание с повторениями, и наоборот -- любому сочетанию соответствует такая последовательность. 

		Общее число символов в последовательности: $n + k - 1$. Из них выберем $k$ позиций для звезд (или $n-1$ для перегородок). Тогда таких последовательностей:
		$$
		\overline{C_n^k} = \frac{(n+k-1)!}{n!(k-1)!}
		$$
	\end{prbox}
\end{document}