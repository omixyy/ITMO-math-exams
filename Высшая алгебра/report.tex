\documentclass[14pt]{extreport}
\usepackage{graphicx}
\usepackage{gost}
\usepackage[T1,T2A]{fontenc}
\usepackage[utf8]{inputenc}
\PassOptionsToPackage{russian,english}{babel}
\usepackage[english,russian]{babel}
\usepackage{tempora}
\usepackage{hyperref}
\linespread{1.3}
\setlength{\parindent}{1.25cm}
\usepackage{hyperref}

\usepackage{multirow}
\usepackage{makecell}
\usepackage{longtable}
\usepackage{array}
\newcolumntype{P}[1]{>{\raggedright\arraybackslash}p{#1}}

\usepackage{fancyhdr}
\pagestyle{fancy}
\fancyhf{} 
\fancyhead{} 
\fancyfoot[C]{\thepage} 

\usepackage{amsmath, amssymb}
\usepackage{tcolorbox}
\usepackage{enumitem}
\usepackage{geometry}
\usepackage{amsthm}

\newtcolorbox{thbox}[1][]{colback=orange!10, colframe=orange!60, title=Теорема, #1}
\newtcolorbox{prbox}[1][]{colback=green!5, colframe=green!60!gray, title=Доказательство, #1}
\newtcolorbox{defbox}[1][]{colback=blue!5, colframe=blue!60, title=Определение, #1}
\newtcolorbox{impbox}[1][]{colback=red!5, colframe=red!60, title=Важно, #1}

\newcommand{\defeq}[0]{\stackrel{\text{def}}{=}}

\begin{document}
\thispagestyle{empty}
	\title{Высшая алгебра. Экзамен.}
	\maketitle

	\newpage
	\tableofcontents

    \newpage
    \chapter{Матрица. Определение сложения матриц, умножения матрицы на число, свойства}
    \section{Матрица}

    \begin{defbox}
        \textbf{Матрица} -- прямоугольная таблица чисел $m \times n$ 
    \end{defbox}

    \section{Действия над матрицами}

    \begin{defbox}
        \textbf{Сложение матриц} -- операция над матрицами, при которой складываются их соответственные элементы. Для матриц разных размеров операция не определена.
        $$
        \begin{pmatrix}
            1 & 0 & 0\\
            0 & 1 & 1\\
            1 & 1 & 0
        \end{pmatrix}
        +
        \begin{pmatrix}
            0 & 1 & 1\\
            1 & 0 & 0\\
            0 & 0 & 1
        \end{pmatrix}
        =\begin{pmatrix}
            1 & 1 & 1\\
            1 & 1 & 1\\
            1 & 1 & 1
        \end{pmatrix}
        $$
    \end{defbox}

    \begin{defbox}
        \textbf{Умножение матрицы на число} -- операция, при которой каждый элемент матрицы умножается на данное число.

        $$
        \lambda \begin{pmatrix}
            1 & 1\\
            0 & 1
        \end{pmatrix}
        =
        \begin{pmatrix}
            \lambda & \lambda \\
            0 & \lambda
        \end{pmatrix}
        $$
    \end{defbox}

    \chapter{Определение и свойства операции умножения матриц}
    \section{Умножение матриц}
    \begin{defbox}
        \textbf{Умножение матрицы на матрицу} -- это операция над двумя матрицами, при которой каждый $i$-й элемент $n$-й строки первой матрицы умножается на $i$-ый элемент $n$-го столбца второй матрицы, после чего все такие произведения суммируются
        $$
        A \cdot B = C
        $$, где 
        $$
        c_{ij} = \sum_{t=1}^{k}a_{it}b_{tj}
        $$
    \end{defbox}
    \section{Свойства умножения матриц}
    \begin{enumerate}
        \item $A(BC) = (AB)C$
        \item $A(B+C) = AB+AC$
        \item $(B+C)D = BD + CD$
    \end{enumerate}

    \begin{impbox}
        Умножение матриц не коммутативно!
        $$
        AB \ne BA
        $$
    \end{impbox}

    \chapter{Транспонирование матрицы. Свойства}

    \begin{defbox}
        \textbf{Транспонирование матрицы} -- процесс преобразование матрицы, при котором столбцы становятся строками, а строки -- столбцами.
        $$
        A = \begin{pmatrix}
            0 & 1 & 1\\
            1 & 0 & 0\\
            0 & 0 & 1
        \end{pmatrix}
        \Leftrightarrow
        A^T = \begin{pmatrix}
            0 & 1 & 0\\
            1 & 0 & 0\\
            1 & 0 & 1
        \end{pmatrix}
        $$
    \end{defbox}

    \noindent Свойства транспонирования матрицы:
    \begin{enumerate}
        \item $(A^T)^T=A$
        \item $A^T+B^T=(A+B)^T$
        \item $\lambda A^T = (\lambda A)^T$
        \item $(AB)^T=B^TA^T$
        \item $(A^{-1})^T= (A^T)^{-1}$
        \item $\det A = \det A^T$
        \item $\operatorname{rang} A = \operatorname{rang} A^T$
    \end{enumerate}

    \chapter{Отдельные виды матриц: квадратная, треугольная, диагональная, единичная. Определения и свойства}

    \section{Виды матриц}

    \begin{defbox}
        Матрица называется \textbf{квадратной}, если число ее строк равно числу ее столбцов
        $$
        A = \begin{pmatrix}
            0 & 1 & 0\\
            1 & 0 & 0\\
            1 & 0 & 1
        \end{pmatrix}
        $$
    \end{defbox}

    \noindent Свойства квадратной матрицы:
    \begin{enumerate}
        \item Можно вычислить определитель
        \item Может быть обратимой, если $\det A \ne 0$
        \item Можно возводить в целую степень
    \end{enumerate}
    \newpage

    \begin{defbox}
        Матрица называется \textbf{верхнетреугольной}, если она является квадратной, а \textbf{под} ее главной диагональю все элементы равны нулю
        $$
        A = \begin{pmatrix}
            0 & 1 & 2\\
            0 & 3 & 1\\
            0 & 0 & 1
        \end{pmatrix}
        $$
    \end{defbox}

    \begin{defbox}
        Матрица называется \textbf{нижнетреугольной}, если она является квадратной, а \textbf{над} ее главной диагональю все элементы равны нулю
        $$
        A = \begin{pmatrix}
            0 & 0 & 0\\
            2 & 3 & 0\\
            1 & 0 & 1
        \end{pmatrix}
        $$
    \end{defbox}

    \noindent Свойства треугольных матриц:
    \begin{enumerate}
        \item Определитель равен произведению диагональных элементов
        \item Обратная к треугольной матрице (если существует) тоже треугольная того же типа
        \item Произведение двух верхних (нижних) треугольных матриц — верхняя (нижняя) треугольная матрица
    \end{enumerate}

    \newpage

    \begin{defbox}
        Матрица называется \textbf{диагональной}, если все ее элементы, кроме тех, что лежат на главной диагонали, равны нулю
        $$
        A = \begin{pmatrix}
            2 & 0 & 0\\
            0 & 3 & 0\\
            0 & 0 & 1
        \end{pmatrix}
        $$
    \end{defbox}
    \noindent Свойства диагональной матрицы:
    \begin{enumerate}
        \item Определитель равен произведению диагональных элементов
        \item Обратная: $D^{-1}=\operatorname{diag} \left( d_1^{-1}, d_2^{-1}, ...\right)$
        \item Возведение в степень: $D^k=\operatorname{diag} \left( d_1^k, d_2^k, ...\right)$
    \end{enumerate}
    \begin{defbox}
        Матрица называется \textbf{единичной}, если все элементы ее главной диагонали равны единице, а остальные -- нулю
        $$
        E = \begin{pmatrix}
            1 & 0 & 0\\
            0 & 1 & 0\\
            0 & 0 & 1
        \end{pmatrix}
        $$
    \end{defbox}
    \noindent Свойства единичной матрицы:
    \begin{enumerate}
        \item Нейтральный элемент относительно умножения
        \item $\det E = 1$
        \item $E^{-1}=E$
    \end{enumerate}

    \chapter{Определение многочлена от матрицы.}

    \begin{defbox}
        Пусть дан полином $p(x)$ с коэффициентами из поля $\mathbb{K}$:
        $$
        p(x) = a_0+a_1x+a_2x^2+...+a_nx^n
        $$
        где $a_1, a_2...a_n \in \mathbb{K}$

        Пусть $A$ -- квадратная матрица размера $m\times m$. Тогда \textbf{многочленом от матрицы} $p(A)$ называется матрица, получаемая формальной подстановкой матрица $A$ вместо переменной $x$ в выражение для многочлена.
        $$
        p(A) = a_0E + a_1A + a_2A^2+...+a_nA^n
        $$
    \end{defbox}

    \chapter{Определители 2-го и 3-го порядка. Определения. Правила вычисления.}

    \section{Определение}
    \begin{defbox}
        \textbf{Определитель} — это число, которое ставится в соответствие квадратной матрице и вычисляется по определённым правилам. 
    \end{defbox}

    \section{Правила вычисления}
    $$
    \begin{vmatrix}
        a_{11} & a_{12}\\
        a_{21} & a_{22}
    \end{vmatrix}
    = a_{11}a_{22} - a_{12}a_{21}
    $$
    $$
    \begin{aligned}
    \begin{vmatrix}
        a_{11} & a_{12} & a_{13}\\
        a_{21} & a_{22} & a_{23}\\
        a_{31} & a_{32} & a_{33}
    \end{vmatrix}
    &= a_{11}a_{22}a_{33} + a_{21}a_{32}a_{13} + a_{12}a_{23}a_{31} \\
    &\quad - a_{31}a_{22}a_{11} - a_{32}a_{23}a_{11} - a_{21}a_{12}a_{33}
    \end{aligned}
    $$

    \chapter{Определитель матрицы порядка n. Определение через рекуррентное разложение. Вычисление с применением разложения по строке (столбцу).}

    \section{Определения}
    \begin{defbox}
        \textbf{Определитель матрицы порядка $n$} -- число, которое ставится в соответствие этой матрице и вычисляется по определённым правилам.
    \end{defbox}

    Разложение по $i$-й строке:
    $$
    \det A = \sum_{j=1}^{n}a_{ij}A_{ij}
    $$
    где $A_{ij}$ -- алгебраическое дополнение элемента с индексом $ij$ в матрице $A$.

    $$
    A_{ij} = (-1)^{i+j}M_{ij}
    $$
    где $M_{ij}$ -- минор элемента $ij$ матрицы $A$ 

    \begin{defbox}
        \textbf{Минор элемента $ij$ матрицы $A$} -- определитель матрицы $A$, получающийся путем вычеркивания из матрицы $i$-й строки и $j$-того столбца
    \end{defbox}

    \section{Пример вычисления}
    Вычислим определитель матрицы $4$-го порядка:

\[
\Delta = 
\begin{vmatrix}
2 & 0 & 1 & -1 \\
3 & 0 & 0 & 4 \\
-1 & 2 & 2 & -3 \\
1 & 1 & 1 & 0
\end{vmatrix}.
\]

\subsection*{Шаг 1: Выбор строки для разложения}
Выберем вторую строку $(3, 0, 0, 4)$, так как в ней два нулевых элемента. Это упростит вычисления.

Формула разложения по $i$-й строке:
\[
\Delta = a_{i1}A_{i1} + a_{i2}A_{i2} + a_{i3}A_{i3} + a_{i4}A_{i4},
\]
где $A_{ij} = (-1)^{i+j}M_{ij}$ -- алгебраическое дополнение элемента $a_{ij}$.

Для $i=2$:
\[
\Delta = 3 \cdot A_{21} + 0 \cdot A_{22} + 0 \cdot A_{23} + 4 \cdot A_{24} = 3A_{21} + 4A_{24}.
\]

\subsection*{Шаг 2: Вычисление алгебраических дополнений}

\textbf{1. Вычисляем $A_{21} = (-1)^{2+1} M_{21}$:}

\[
M_{21} = 
\begin{vmatrix}
0 & 1 & -1 \\
2 & 2 & -3 \\
1 & 1 & 0
\end{vmatrix}.
\]

Вычислим этот определитель $3$-го порядка разложением по первой строке:
\begin{align*}
M_{21} &= 0 \cdot \begin{vmatrix} 2 & -3 \\ 1 & 0 \end{vmatrix} 
        - 1 \cdot \begin{vmatrix} 2 & -3 \\ 1 & 0 \end{vmatrix} 
        + (-1) \cdot \begin{vmatrix} 2 & 2 \\ 1 & 1 \end{vmatrix} \\
       &= 0 \cdot (2\cdot0 - (-3)\cdot1) 
          - 1 \cdot (2\cdot0 - (-3)\cdot1) 
          - 1 \cdot (2\cdot1 - 2\cdot1) \\
       &= 0 \cdot 3 - 1 \cdot 3 - 1 \cdot 0 \\
       &= -3.
\end{align*}

Следовательно, $A_{21} = (-1)^{3} \cdot (-3) = (-1) \cdot (-3) = 3$.

\textbf{2. Вычисляем $A_{24} = (-1)^{2+4} M_{24}$:}

\[
M_{24} = 
\begin{vmatrix}
2 & 0 & 1 \\
-1 & 2 & 2 \\
1 & 1 & 1
\end{vmatrix}.
\]

Вычислим этот определитель $3$-го порядка:
\begin{align*}
M_{24} &= 2 \cdot \begin{vmatrix} 2 & 2 \\ 1 & 1 \end{vmatrix} 
        - 0 \cdot \begin{vmatrix} -1 & 2 \\ 1 & 1 \end{vmatrix} 
        + 1 \cdot \begin{vmatrix} -1 & 2 \\ 1 & 1 \end{vmatrix} \\
       &= 2 \cdot (2\cdot1 - 2\cdot1) 
          - 0 \cdot ((-1)\cdot1 - 2\cdot1) 
          + 1 \cdot ((-1)\cdot1 - 2\cdot1) \\
       &= 2 \cdot 0 - 0 \cdot (-3) + 1 \cdot (-3) \\
       &= -3.
\end{align*}

Следовательно, $A_{24} = (-1)^{6} \cdot (-3) = 1 \cdot (-3) = -3$.

\subsection*{Шаг 3: Подставляем в формулу разложения}
\[
\Delta = 3 \cdot A_{21} + 4 \cdot A_{24} = 3 \cdot 3 + 4 \cdot (-3) = 9 - 12 = -3.
\]

\subsection*{Ответ:}
\[
\boxed{\Delta = -3}
\]

    \chapter{Свойства определителя: определитель транспонированной матрицы, определитель произведения матриц}

    \noindent Свойства определителя:
    \begin{enumerate}
        \item Определитель транспонированной матрицы: $\det A = \det A^T$ (разложение по столбцу дает тот же результат, что и разложение по строке)
        \item Определитель произведения матриц: $\det AB = \det A \det B$
        \item Определитель меняет знак при перестановке двух строк (столбцов) местами
        \item Определитель, у которого есть две равные строки (столбца), равен нулю
        \item Определитель не изменится, если к любой строке (столбцу) прибавить другую строку (столбец)
    \end{enumerate}

    \chapter{Свойства определителя, связанные с линейной комбинацией строк (столбцов).}

    \noindent Свойства определителя:
    \begin{enumerate}
        \item Если матрица $A$ получена из матрицы $B$ умножением $i$-й строки на $k$: $\det A = k\det B$
        \item Если $i$-я строка матрицы является суммой двух строк $u$ и $v$, то определитель равен сумме определителей двух матриц, в которых эта строка заменена на $u$ и $v$ соответственно: 
        $$
        \begin{vmatrix}
            a_{11} & a_{12}\\
            u_1+v_1 & u_2+v_2
        \end{vmatrix}
        =
        \begin{vmatrix}
            a_{11} & a_{12}\\
            u_1 & u_2
        \end{vmatrix}
        +
        \begin{vmatrix}
            a_{11} & a_{12}\\
            v_1 & v_2
        \end{vmatrix}
        $$
        \item Если к какой-либо строки (столбцу) прибавить другую строку (столбец), то определитель не измменится
    \end{enumerate} 

    \chapter{Изменение определителя при элементарных преобразованиях матрицы}
    \begin{enumerate}
        \item При перестановке любых двух строк (столбцов) определитель меняет знак на противоположный
        \item При умножении любой строки (столбца) на ненулевое число определитель также умножается на это число
        \item При прибавлении одной строки, умноженной на число, к другой, определитель не меняется
    \end{enumerate}

    \chapter{Свойства определителя: определитель треугольной матрицы, определитель диагональной матрицы}
    \begin{enumerate}
        \item Определитель треугольной матрицы -- произведение диагональных элементов
        \item Определитель диагональной матрицы -- произведение диагональных элементов
    \end{enumerate}

    \chapter{Вычисление определителя методом приведения к треугольному виду.}

    Метод заключается в приведении матрицы к треугольному виду, отслеживая при этом изменения определителя при проведении элементарных преобразований.

    Дана матрица $A$ размера $3 \times 3$:
\[
A = \begin{pmatrix}
     2 &  3 & -1 \\
     4 &  1 &  3 \\
    -2 &  2 &  1
    \end{pmatrix}
\]

Найти $\det(A)$.

\subsection*{Решение}

Применяем элементарные преобразования строк, не меняющие определитель
(прибавление к одной строке другой, умноженной на число):

\begin{enumerate}
    \item Обнулим элементы под $a_{11} = 2$:
    \[
    \begin{vmatrix}
     2 &  3 & -1 \\
     4 &  1 &  3 \\
    -2 &  2 &  1
    \end{vmatrix}
    =
    \begin{vmatrix}
     2 &  3 & -1 \\
     0 & -5 &  5 \\
     0 &  5 &  0
    \end{vmatrix}
    \]
    где:
    \begin{itemize}
        \item $R_2 \gets R_2 - 2R_1$
        \item $R_3 \gets R_3 + R_1$ (так как $-2/2 = -1$, то $R_3 \gets R_3 - (-1)R_1$)
    \end{itemize}

    \item Обнулим элемент под $a_{22} = -5$:
    \[
    \begin{vmatrix}
     2 &  3 & -1 \\
     0 & -5 &  5 \\
     0 &  5 &  0
    \end{vmatrix}
    =
    \begin{vmatrix}
     2 &  3 & -1 \\
     0 & -5 &  5 \\
     0 &  0 &  5
    \end{vmatrix}
    \]
    где $R_3 \gets R_3 + R_2$ (так как $5/(-5) = -1$, то $R_3 \gets R_3 - (-1)R_2$)
\end{enumerate}

Получили верхнюю треугольную матрицу $U$:
\[
U = \begin{pmatrix}
     2 &  3 & -1 \\
     0 & -5 &  5 \\
     0 &  0 &  5
    \end{pmatrix}
\]

\subsection*{Вычисление определителя}

Определитель треугольной матрицы равен произведению элементов главной диагонали:
\[
\det(U) = 2 \cdot (-5) \cdot 5 = -50
\]

Так как мы использовали только преобразования вида $R_i \gets R_i + \lambda R_j$,
определитель не изменился:
\[
\boxed{\det(A) = -50}
\]

\section*{Пример с перестановкой строк}

Рассмотрим матрицу:
\[
B = \begin{pmatrix}
     0 &  2 &  1 \\
     1 &  2 &  3 \\
     4 &  5 &  6
    \end{pmatrix}
\]

\begin{enumerate}
    \item $b_{11} = 0$ -- нужна перестановка. Меняем $R_1 \leftrightarrow R_2$:
    \[
    \det(B) = -\begin{vmatrix}
     1 &  2 &  3 \\
     0 &  2 &  1 \\
     4 &  5 &  6
    \end{vmatrix}
    \]
    (знак определителя изменился)

    \item Обнуляем первый столбец:
    \[
    R_3 \gets R_3 - 4R_1:\quad
    -\begin{vmatrix}
     1 &  2 &  3 \\
     0 &  2 &  1 \\
     0 & -3 & -6
    \end{vmatrix}
    \]

    \item Обнуляем второй столбец:
    \[
    R_3 \gets R_3 + \frac{3}{2}R_2:\quad
    -\begin{vmatrix}
     1 &  2 &  3 \\
     0 &  2 &  1 \\
     0 &  0 & -\frac{9}{2}
    \end{vmatrix}
    \]
\end{enumerate}

Определитель треугольной матрицы:
\[
1 \cdot 2 \cdot \left(-\frac{9}{2}\right) = -9
\]

Учитывая знак от перестановки:
\[
\det(B) = -(-9) = \boxed{9}
\]

    \chapter{Алгебраическое дополнение элемента матрицы}

    \begin{defbox}
        \textbf{Алгебраическое дополнение элемента матрицы} -- это число, вычисляемое как определитель матрицы, получающейся вычеркиванием строки и столбца, в которых лежит элемент, из исходной матрицы, взятый с положительным знаком, если индекс строки элемента в сумме с его столбцом дает четное число, иначе -- с отрицательным.
        $$
        A_{ij} = (-1)^{i + j}M_{ij}
        $$
    \end{defbox}

    \chapter{Минор, соответствующий элементу матрицы. Связь между минором и алгебраическим дополнением элемента матрицы.}

    \begin{defbox}
        \textbf{Минор элемента $a_{ij}$ матрицы $A$} -- это определитель матрицы, получающийся вычеркиванием из матрицы $i$-й строки и $j$-го столбца
    \end{defbox}

    Связь между алгебраическим дополнением и минором состоит в том, что алгебраическое дополнение элемента $a_{ij}$ и есть минор, взятый со знаком, определяющемся как $(-1)^{i+j}$

    \chapter{Определение обратной матрицы. Свойства обратной матрицы.}

    \begin{defbox}
        \textbf{Обратная матрица к матрице $A$} -- такая матрица $A^{-1}$, которая при умножении на исходную матрицу $A$ дает единичную матрицу $E$
    \end{defbox}

    \noindent Свойства:
    \begin{enumerate}
        \item $AA^{-1} = A^{-1}A=E$
        \item $(A^{-1})^{-1}=A$
        \item $(AB)^{-1}=B^{-1}A^{-1}$
        \item $(A^{-1})^T=(A^T)^{-1}$
        \item $\det A = \frac{1}{\det A^{-1}}$
    \end{enumerate}

    \chapter{Условие обратимости матрицы.}

    \begin{thbox}
        Матрица обратима тогда и только тогда, когда ее определитель не равен нулю.
        $$
        \exists A^{-1} \Leftrightarrow \det A \ne 0
        $$
    \end{thbox}

    \begin{prbox}
    {[Необходимое условие]}

    $$
    AA^{-1}=E \Leftrightarrow \det AA^{-1}=\det E \Leftrightarrow \det A \det A^{-1} = \det E
    $$
    Если $\det A = 0$, то $0 \cdot \det A^{-1}=1$ -- противоречие.
    
    \vspace{20pt}
    {[Достаточное условие]}
    
    Пусть есть $A^{-1}$, $\det A \ne 0$. Проверим, что $AA^{-1}=E$, для этого построим обратную матрицу с помощью присоединений.

    Находим $A_{ij}$ ко всем элементам матрицы $A$, получаем союзную матрицу.
    
    $$
    \widetilde{A} = \begin{pmatrix}
        A_{11} & ... & A_{1n}\\
        ... & A_{ij} & ...\\
        A_{n1} & ... & A_{nn}
    \end{pmatrix}
    $$
    Транспонируем ее, получаем присоединенную: $\hat{A} = \widetilde{A}^T$.
    
    Умножим исходную матрицу на присоединенную:

    $$
    A\hat{A} = \begin{pmatrix}
        a_{11} & ... & a_{1n}\\
        ... & a_{ij} & ...\\
        a_{n1} & ... & a_{nn}
    \end{pmatrix}\begin{pmatrix}
        A_{11} & ... & A_{n1}\\
        ... & A_{ij} & ...\\
        A_{1n} & ... & A_{nn}
    \end{pmatrix}
    =
    \begin{pmatrix}
        \det A & ... & 0\\
        ... & \det A & ...\\
        0 & ... & \det A
    \end{pmatrix}
    $$

    Чтобы получить единичную матрицу, осталось поделить полученное произведение на $\det A$.
    $$
    A^{-1}=\frac{1}{\det A}\hat A
    $$

    \end{prbox}

    \chapter{Построение обратной матрицы с помощью присоединенной матрицы.}

    \noindent Алгоритм построения обратной матрицы:
    \begin{enumerate}
        \item Получение союзной матрицы (матрицы алгебраических дополнений)
        \item Транспонирование союзной матрицы и получение присоединенной
        \item Деление каждого элементаприсоединенной матрицы на определитель исходной
    \end{enumerate}

    $$
    A^{-1}=\frac{1}{\det A}\hat A
    $$

    \chapter{Построение обратной матрицы с помощью элементарных преобразований.}

    Метод состоит в присоединении единичной матрицы к исходной справа и дальнейшем ее приведении к единичной. После этих операций справа появится обратная матрица.

    \chapter{Два определения ранга матрицы: максимальное число линейно независимых строк и размерность наибольшего ненулевого минора.}
     
\end{document}