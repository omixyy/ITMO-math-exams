\documentclass[14pt]{extreport}
\usepackage{graphicx}
\usepackage{gost}
\usepackage[T1,T2A]{fontenc}
\usepackage[utf8]{inputenc}
\PassOptionsToPackage{russian,english}{babel}
\usepackage[english,russian]{babel}
\usepackage{tempora}
\usepackage{hyperref}
\linespread{1.3}
\setlength{\parindent}{1.25cm}
\usepackage{hyperref}

\usepackage{multirow}
\usepackage{makecell}
\usepackage{longtable}
\usepackage{array}
\newcolumntype{P}[1]{>{\raggedright\arraybackslash}p{#1}}

\usepackage{fancyhdr}
\pagestyle{fancy}
\fancyhf{} 
\fancyhead{} 
\fancyfoot[C]{\thepage} 

\usepackage{amsmath, amssymb}
\usepackage{tcolorbox}
\usepackage{enumitem}
\usepackage{geometry}
\usepackage{amsthm}
\tcbuselibrary{breakable}

\newtcolorbox{thbox}[1][]{colback=orange!10, colframe=orange!60, title=Теорема, #1, breakable=true, toprule at break=0pt, % верхняя рамка не показывается при разрыве
    bottomrule at break=0pt}
\newtcolorbox{prbox}[1][]{colback=green!5, colframe=green!60!gray, title=Доказательство, breakable=true, #1, toprule at break=0pt, % верхняя рамка не показывается при разрыве
    bottomrule at break=0pt}
\newtcolorbox{defbox}[1][]{colback=blue!5, colframe=blue!60, title=Определение, #1, breakable=true, toprule at break=0pt, % верхняя рамка не показывается при разрыве
    bottomrule at break=0pt}
\newtcolorbox{impbox}[1][]{colback=red!5, colframe=red!60, title=Важно, #1, breakable=true, toprule at break=0pt, % верхняя рамка не показывается при разрыве
    bottomrule at break=0pt}

\newcommand{\defeq}[0]{\stackrel{\text{def}}{=}}

\begin{document}
\thispagestyle{empty}
	\title{Высшая алгебра. Экзамен.}
	\maketitle

	\newpage
	\tableofcontents

    \newpage
    \chapter{Матрица. Определение сложения матриц, умножения матрицы на число, свойства}
    \section{Матрица}

    \begin{defbox}
        \textbf{Матрица} -- прямоугольная таблица чисел $m \times n$ 
    \end{defbox}

    \section{Действия над матрицами}

    \begin{defbox}
        \textbf{Сложение матриц} -- операция над матрицами, при которой складываются их соответственные элементы. Для матриц разных размеров операция не определена.
        $$
        \begin{pmatrix}
            1 & 0 & 0\\
            0 & 1 & 1\\
            1 & 1 & 0
        \end{pmatrix}
        +
        \begin{pmatrix}
            0 & 1 & 1\\
            1 & 0 & 0\\
            0 & 0 & 1
        \end{pmatrix}
        =\begin{pmatrix}
            1 & 1 & 1\\
            1 & 1 & 1\\
            1 & 1 & 1
        \end{pmatrix}
        $$
    \end{defbox}

    \begin{defbox}
        \textbf{Умножение матрицы на число} -- операция, при которой каждый элемент матрицы умножается на данное число.

        $$
        \lambda \begin{pmatrix}
            1 & 1\\
            0 & 1
        \end{pmatrix}
        =
        \begin{pmatrix}
            \lambda & \lambda \\
            0 & \lambda
        \end{pmatrix}
        $$
    \end{defbox}

    \chapter{Определение и свойства операции умножения матриц}
    \section{Умножение матриц}
    \begin{defbox}
        \textbf{Умножение матрицы на матрицу} -- это операция над двумя матрицами, при которой каждый $i$-й элемент $n$-й строки первой матрицы умножается на $i$-ый элемент $n$-го столбца второй матрицы, после чего все такие произведения суммируются
        $$
        A \cdot B = C
        $$, где 
        $$
        c_{ij} = \sum_{t=1}^{k}a_{it}b_{tj}
        $$
    \end{defbox}
    \section{Свойства умножения матриц}
    \begin{enumerate}
        \item $A(BC) = (AB)C$
        \item $A(B+C) = AB+AC$
        \item $(B+C)D = BD + CD$
    \end{enumerate}

    \begin{impbox}
        Умножение матриц не коммутативно!
        $$
        AB \ne BA
        $$
    \end{impbox}

    \chapter{Транспонирование матрицы. Свойства}

    \begin{defbox}
        \textbf{Транспонирование матрицы} -- процесс преобразование матрицы, при котором столбцы становятся строками, а строки -- столбцами.
        $$
        A = \begin{pmatrix}
            0 & 1 & 1\\
            1 & 0 & 0\\
            0 & 0 & 1
        \end{pmatrix}
        \Leftrightarrow
        A^T = \begin{pmatrix}
            0 & 1 & 0\\
            1 & 0 & 0\\
            1 & 0 & 1
        \end{pmatrix}
        $$
    \end{defbox}

    \noindent Свойства транспонирования матрицы:
    \begin{enumerate}
        \item $(A^T)^T=A$
        \item $A^T+B^T=(A+B)^T$
        \item $\lambda A^T = (\lambda A)^T$
        \item $(AB)^T=B^TA^T$
        \item $(A^{-1})^T= (A^T)^{-1}$
        \item $\det A = \det A^T$
        \item $\operatorname{rang} A = \operatorname{rang} A^T$
    \end{enumerate}

    \chapter{Отдельные виды матриц: квадратная, треугольная, диагональная, единичная. Определения и свойства}

    \section{Виды матриц}

    \begin{defbox}
        Матрица называется \textbf{квадратной}, если число ее строк равно числу ее столбцов
        $$
        A = \begin{pmatrix}
            0 & 1 & 0\\
            1 & 0 & 0\\
            1 & 0 & 1
        \end{pmatrix}
        $$
    \end{defbox}

    \noindent Свойства квадратной матрицы:
    \begin{enumerate}
        \item Можно вычислить определитель
        \item Может быть обратимой, если $\det A \ne 0$
        \item Можно возводить в целую степень
    \end{enumerate}
    \newpage

    \begin{defbox}
        Матрица называется \textbf{верхнетреугольной}, если она является квадратной, а \textbf{под} ее главной диагональю все элементы равны нулю
        $$
        A = \begin{pmatrix}
            0 & 1 & 2\\
            0 & 3 & 1\\
            0 & 0 & 1
        \end{pmatrix}
        $$
    \end{defbox}

    \begin{defbox}
        Матрица называется \textbf{нижнетреугольной}, если она является квадратной, а \textbf{над} ее главной диагональю все элементы равны нулю
        $$
        A = \begin{pmatrix}
            0 & 0 & 0\\
            2 & 3 & 0\\
            1 & 0 & 1
        \end{pmatrix}
        $$
    \end{defbox}

    \noindent Свойства треугольных матриц:
    \begin{enumerate}
        \item Определитель равен произведению диагональных элементов
        \item Обратная к треугольной матрице (если существует) тоже треугольная того же типа
        \item Произведение двух верхних (нижних) треугольных матриц — верхняя (нижняя) треугольная матрица
    \end{enumerate}

    \newpage

    \begin{defbox}
        Матрица называется \textbf{диагональной}, если все ее элементы, кроме тех, что лежат на главной диагонали, равны нулю
        $$
        A = \begin{pmatrix}
            2 & 0 & 0\\
            0 & 3 & 0\\
            0 & 0 & 1
        \end{pmatrix}
        $$
    \end{defbox}
    \noindent Свойства диагональной матрицы:
    \begin{enumerate}
        \item Определитель равен произведению диагональных элементов
        \item Обратная: $D^{-1}=\operatorname{diag} \left( d_1^{-1}, d_2^{-1}, ...\right)$
        \item Возведение в степень: $D^k=\operatorname{diag} \left( d_1^k, d_2^k, ...\right)$
    \end{enumerate}
    \begin{defbox}
        Матрица называется \textbf{единичной}, если все элементы ее главной диагонали равны единице, а остальные -- нулю
        $$
        E = \begin{pmatrix}
            1 & 0 & 0\\
            0 & 1 & 0\\
            0 & 0 & 1
        \end{pmatrix}
        $$
    \end{defbox}
    \noindent Свойства единичной матрицы:
    \begin{enumerate}
        \item Нейтральный элемент относительно умножения
        \item $\det E = 1$
        \item $E^{-1}=E$
    \end{enumerate}

    \chapter{Определение многочлена от матрицы.}

    \begin{defbox}
        Пусть дан полином $p(x)$ с коэффициентами из поля $\mathbb{K}$:
        $$
        p(x) = a_0+a_1x+a_2x^2+...+a_nx^n
        $$
        где $a_1, a_2...a_n \in \mathbb{K}$

        Пусть $A$ -- квадратная матрица размера $m\times m$. Тогда \textbf{многочленом от матрицы} $p(A)$ называется матрица, получаемая формальной подстановкой матрица $A$ вместо переменной $x$ в выражение для многочлена.
        $$
        p(A) = a_0E + a_1A + a_2A^2+...+a_nA^n
        $$
    \end{defbox}

    \chapter{Определители 2-го и 3-го порядка. Определения. Правила вычисления.}

    \section{Определение}
    \begin{defbox}
        \textbf{Определитель} — это число, которое ставится в соответствие квадратной матрице и вычисляется по определённым правилам. 
    \end{defbox}

    \section{Правила вычисления}
    $$
    \begin{vmatrix}
        a_{11} & a_{12}\\
        a_{21} & a_{22}
    \end{vmatrix}
    = a_{11}a_{22} - a_{12}a_{21}
    $$
    $$
    \begin{aligned}
    \begin{vmatrix}
        a_{11} & a_{12} & a_{13}\\
        a_{21} & a_{22} & a_{23}\\
        a_{31} & a_{32} & a_{33}
    \end{vmatrix}
    &= a_{11}a_{22}a_{33} + a_{21}a_{32}a_{13} + a_{12}a_{23}a_{31} \\
    &\quad - a_{31}a_{22}a_{11} - a_{32}a_{23}a_{11} - a_{21}a_{12}a_{33}
    \end{aligned}
    $$

    \chapter{Определитель матрицы порядка n. Определение через рекуррентное разложение. Вычисление с применением разложения по строке (столбцу).}

    \section{Определения}
    \begin{defbox}
        \textbf{Определитель матрицы порядка $n$} -- число, которое ставится в соответствие этой матрице и вычисляется по определённым правилам.
    \end{defbox}

    Разложение по $i$-й строке:
    $$
    \det A = \sum_{j=1}^{n}a_{ij}A_{ij}
    $$
    где $A_{ij}$ -- алгебраическое дополнение элемента с индексом $ij$ в матрице $A$.

    $$
    A_{ij} = (-1)^{i+j}M_{ij}
    $$
    где $M_{ij}$ -- минор элемента $ij$ матрицы $A$ 

    \begin{defbox}
        \textbf{Минор элемента $ij$ матрицы $A$} -- определитель матрицы $A$, получающийся путем вычеркивания из матрицы $i$-й строки и $j$-того столбца
    \end{defbox}

    \section{Пример вычисления}
    Вычислим определитель матрицы $4$-го порядка:

\[
\Delta = 
\begin{vmatrix}
2 & 0 & 1 & -1 \\
3 & 0 & 0 & 4 \\
-1 & 2 & 2 & -3 \\
1 & 1 & 1 & 0
\end{vmatrix}.
\]

\subsection*{Шаг 1: Выбор строки для разложения}
Выберем вторую строку $(3, 0, 0, 4)$, так как в ней два нулевых элемента. Это упростит вычисления.

Формула разложения по $i$-й строке:
\[
\Delta = a_{i1}A_{i1} + a_{i2}A_{i2} + a_{i3}A_{i3} + a_{i4}A_{i4},
\]
где $A_{ij} = (-1)^{i+j}M_{ij}$ -- алгебраическое дополнение элемента $a_{ij}$.

Для $i=2$:
\[
\Delta = 3 \cdot A_{21} + 0 \cdot A_{22} + 0 \cdot A_{23} + 4 \cdot A_{24} = 3A_{21} + 4A_{24}.
\]

\subsection*{Шаг 2: Вычисление алгебраических дополнений}

\textbf{1. Вычисляем $A_{21} = (-1)^{2+1} M_{21}$:}

\[
M_{21} = 
\begin{vmatrix}
0 & 1 & -1 \\
2 & 2 & -3 \\
1 & 1 & 0
\end{vmatrix}.
\]

Вычислим этот определитель $3$-го порядка разложением по первой строке:
\begin{align*}
M_{21} &= 0 \cdot \begin{vmatrix} 2 & -3 \\ 1 & 0 \end{vmatrix} 
        - 1 \cdot \begin{vmatrix} 2 & -3 \\ 1 & 0 \end{vmatrix} 
        + (-1) \cdot \begin{vmatrix} 2 & 2 \\ 1 & 1 \end{vmatrix} \\
       &= 0 \cdot (2\cdot0 - (-3)\cdot1) 
          - 1 \cdot (2\cdot0 - (-3)\cdot1) 
          - 1 \cdot (2\cdot1 - 2\cdot1) \\
       &= 0 \cdot 3 - 1 \cdot 3 - 1 \cdot 0 \\
       &= -3.
\end{align*}

Следовательно, $A_{21} = (-1)^{3} \cdot (-3) = (-1) \cdot (-3) = 3$.

\textbf{2. Вычисляем $A_{24} = (-1)^{2+4} M_{24}$:}

\[
M_{24} = 
\begin{vmatrix}
2 & 0 & 1 \\
-1 & 2 & 2 \\
1 & 1 & 1
\end{vmatrix}.
\]

Вычислим этот определитель $3$-го порядка:
\begin{align*}
M_{24} &= 2 \cdot \begin{vmatrix} 2 & 2 \\ 1 & 1 \end{vmatrix} 
        - 0 \cdot \begin{vmatrix} -1 & 2 \\ 1 & 1 \end{vmatrix} 
        + 1 \cdot \begin{vmatrix} -1 & 2 \\ 1 & 1 \end{vmatrix} \\
       &= 2 \cdot (2\cdot1 - 2\cdot1) 
          - 0 \cdot ((-1)\cdot1 - 2\cdot1) 
          + 1 \cdot ((-1)\cdot1 - 2\cdot1) \\
       &= 2 \cdot 0 - 0 \cdot (-3) + 1 \cdot (-3) \\
       &= -3.
\end{align*}

Следовательно, $A_{24} = (-1)^{6} \cdot (-3) = 1 \cdot (-3) = -3$.

\subsection*{Шаг 3: Подставляем в формулу разложения}
\[
\Delta = 3 \cdot A_{21} + 4 \cdot A_{24} = 3 \cdot 3 + 4 \cdot (-3) = 9 - 12 = -3.
\]

\subsection*{Ответ:}
\[
\boxed{\Delta = -3}
\]

    \chapter{Свойства определителя: определитель транспонированной матрицы, определитель произведения матриц}

    \noindent Свойства определителя:
    \begin{enumerate}
        \item Определитель транспонированной матрицы: $\det A = \det A^T$ (разложение по столбцу дает тот же результат, что и разложение по строке)
        \item Определитель произведения матриц: $\det AB = \det A \det B$
        \item Определитель меняет знак при перестановке двух строк (столбцов) местами
        \item Определитель, у которого есть две равные строки (столбца), равен нулю
        \item Определитель не изменится, если к любой строке (столбцу) прибавить другую строку (столбец)
    \end{enumerate}

    \chapter{Свойства определителя, связанные с линейной комбинацией строк (столбцов).}

    \noindent Свойства определителя:
    \begin{enumerate}
        \item Если матрица $A$ получена из матрицы $B$ умножением $i$-й строки на $k$: $\det A = k\det B$
        \item Если $i$-я строка матрицы является суммой двух строк $u$ и $v$, то определитель равен сумме определителей двух матриц, в которых эта строка заменена на $u$ и $v$ соответственно: 
        $$
        \begin{vmatrix}
            a_{11} & a_{12}\\
            u_1+v_1 & u_2+v_2
        \end{vmatrix}
        =
        \begin{vmatrix}
            a_{11} & a_{12}\\
            u_1 & u_2
        \end{vmatrix}
        +
        \begin{vmatrix}
            a_{11} & a_{12}\\
            v_1 & v_2
        \end{vmatrix}
        $$
        \item Если к какой-либо строки (столбцу) прибавить другую строку (столбец), то определитель не измменится
    \end{enumerate} 

    \chapter{Изменение определителя при элементарных преобразованиях матрицы}
    \begin{enumerate}
        \item При перестановке любых двух строк (столбцов) определитель меняет знак на противоположный
        \item При умножении любой строки (столбца) на ненулевое число определитель также умножается на это число
        \item При прибавлении одной строки, умноженной на число, к другой, определитель не меняется
    \end{enumerate}

    \chapter{Свойства определителя: определитель треугольной матрицы, определитель диагональной матрицы}
    \begin{enumerate}
        \item Определитель треугольной матрицы -- произведение диагональных элементов
        \item Определитель диагональной матрицы -- произведение диагональных элементов
    \end{enumerate}

    \chapter{Вычисление определителя методом приведения к треугольному виду.}

    Метод заключается в приведении матрицы к треугольному виду, отслеживая при этом изменения определителя при проведении элементарных преобразований.

    Дана матрица $A$ размера $3 \times 3$:
\[
A = \begin{pmatrix}
     2 &  3 & -1 \\
     4 &  1 &  3 \\
    -2 &  2 &  1
    \end{pmatrix}
\]

Найти $\det(A)$.

\subsection*{Решение}

Применяем элементарные преобразования строк, не меняющие определитель
(прибавление к одной строке другой, умноженной на число):

\begin{enumerate}
    \item Обнулим элементы под $a_{11} = 2$:
    \[
    \begin{vmatrix}
     2 &  3 & -1 \\
     4 &  1 &  3 \\
    -2 &  2 &  1
    \end{vmatrix}
    =
    \begin{vmatrix}
     2 &  3 & -1 \\
     0 & -5 &  5 \\
     0 &  5 &  0
    \end{vmatrix}
    \]
    где:
    \begin{itemize}
        \item $R_2 \gets R_2 - 2R_1$
        \item $R_3 \gets R_3 + R_1$ (так как $-2/2 = -1$, то $R_3 \gets R_3 - (-1)R_1$)
    \end{itemize}

    \item Обнулим элемент под $a_{22} = -5$:
    \[
    \begin{vmatrix}
     2 &  3 & -1 \\
     0 & -5 &  5 \\
     0 &  5 &  0
    \end{vmatrix}
    =
    \begin{vmatrix}
     2 &  3 & -1 \\
     0 & -5 &  5 \\
     0 &  0 &  5
    \end{vmatrix}
    \]
    где $R_3 \gets R_3 + R_2$ (так как $5/(-5) = -1$, то $R_3 \gets R_3 - (-1)R_2$)
\end{enumerate}

Получили верхнюю треугольную матрицу $U$:
\[
U = \begin{pmatrix}
     2 &  3 & -1 \\
     0 & -5 &  5 \\
     0 &  0 &  5
    \end{pmatrix}
\]

\subsection*{Вычисление определителя}

Определитель треугольной матрицы равен произведению элементов главной диагонали:
\[
\det(U) = 2 \cdot (-5) \cdot 5 = -50
\]

Так как мы использовали только преобразования вида $R_i \gets R_i + \lambda R_j$,
определитель не изменился:
\[
\boxed{\det(A) = -50}
\]

\section*{Пример с перестановкой строк}

Рассмотрим матрицу:
\[
B = \begin{pmatrix}
     0 &  2 &  1 \\
     1 &  2 &  3 \\
     4 &  5 &  6
    \end{pmatrix}
\]

\begin{enumerate}
    \item $b_{11} = 0$ -- нужна перестановка. Меняем $R_1 \leftrightarrow R_2$:
    \[
    \det(B) = -\begin{vmatrix}
     1 &  2 &  3 \\
     0 &  2 &  1 \\
     4 &  5 &  6
    \end{vmatrix}
    \]
    (знак определителя изменился)

    \item Обнуляем первый столбец:
    \[
    R_3 \gets R_3 - 4R_1:\quad
    -\begin{vmatrix}
     1 &  2 &  3 \\
     0 &  2 &  1 \\
     0 & -3 & -6
    \end{vmatrix}
    \]

    \item Обнуляем второй столбец:
    \[
    R_3 \gets R_3 + \frac{3}{2}R_2:\quad
    -\begin{vmatrix}
     1 &  2 &  3 \\
     0 &  2 &  1 \\
     0 &  0 & -\frac{9}{2}
    \end{vmatrix}
    \]
\end{enumerate}

Определитель треугольной матрицы:
\[
1 \cdot 2 \cdot \left(-\frac{9}{2}\right) = -9
\]

Учитывая знак от перестановки:
\[
\det(B) = -(-9) = \boxed{9}
\]

    \chapter{Алгебраическое дополнение элемента матрицы}

    \begin{defbox}
        \textbf{Алгебраическое дополнение элемента матрицы} -- это число, вычисляемое как определитель матрицы, получающейся вычеркиванием строки и столбца, в которых лежит элемент, из исходной матрицы, взятый с положительным знаком, если индекс строки элемента в сумме с его столбцом дает четное число, иначе -- с отрицательным.
        $$
        A_{ij} = (-1)^{i + j}M_{ij}
        $$
    \end{defbox}

    \chapter{Минор, соответствующий элементу матрицы. Связь между минором и алгебраическим дополнением элемента матрицы.}

    \begin{defbox}
        \textbf{Минор элемента $a_{ij}$ матрицы $A$} -- это определитель матрицы, получающийся вычеркиванием из матрицы $i$-й строки и $j$-го столбца
    \end{defbox}

    Связь между алгебраическим дополнением и минором состоит в том, что алгебраическое дополнение элемента $a_{ij}$ и есть минор, взятый со знаком, определяющемся как $(-1)^{i+j}$

    \chapter{Определение обратной матрицы. Свойства обратной матрицы.}

    \begin{defbox}
        \textbf{Обратная матрица к матрице $A$} -- такая матрица $A^{-1}$, которая при умножении на исходную матрицу $A$ дает единичную матрицу $E$
    \end{defbox}

    \noindent Свойства:
    \begin{enumerate}
        \item $AA^{-1} = A^{-1}A=E$
        \item $(A^{-1})^{-1}=A$
        \item $(AB)^{-1}=B^{-1}A^{-1}$
        \item $(A^{-1})^T=(A^T)^{-1}$
        \item $\det A = \frac{1}{\det A^{-1}}$
    \end{enumerate}

    \chapter{Условие обратимости матрицы.}

    \begin{thbox}
        Матрица обратима тогда и только тогда, когда ее определитель не равен нулю.
        $$
        \exists A^{-1} \Leftrightarrow \det A \ne 0
        $$
    \end{thbox}

    \begin{prbox}
    {[Необходимое условие]}

    $$
    AA^{-1}=E \Leftrightarrow \det AA^{-1}=\det E \Leftrightarrow \det A \det A^{-1} = \det E
    $$
    Если $\det A = 0$, то $0 \cdot \det A^{-1}=1$ -- противоречие.
    
    \vspace{20pt}
    {[Достаточное условие]}
    
    Пусть есть $A^{-1}$, $\det A \ne 0$. Проверим, что $AA^{-1}=E$, для этого построим обратную матрицу с помощью присоединений.

    Находим $A_{ij}$ ко всем элементам матрицы $A$, получаем союзную матрицу.
    
    $$
    \widetilde{A} = \begin{pmatrix}
        A_{11} & ... & A_{1n}\\
        ... & A_{ij} & ...\\
        A_{n1} & ... & A_{nn}
    \end{pmatrix}
    $$
    Транспонируем ее, получаем присоединенную: $\hat{A} = \widetilde{A}^T$.
    
    Умножим исходную матрицу на присоединенную:

    $$
    A\hat{A} = \begin{pmatrix}
        a_{11} & ... & a_{1n}\\
        ... & a_{ij} & ...\\
        a_{n1} & ... & a_{nn}
    \end{pmatrix}\begin{pmatrix}
        A_{11} & ... & A_{n1}\\
        ... & A_{ij} & ...\\
        A_{1n} & ... & A_{nn}
    \end{pmatrix}
    =
    \begin{pmatrix}
        \det A & ... & 0\\
        ... & \det A & ...\\
        0 & ... & \det A
    \end{pmatrix}
    $$

    Чтобы получить единичную матрицу, осталось поделить полученное произведение на $\det A$.
    $$
    A^{-1}=\frac{1}{\det A}\hat A
    $$

    \end{prbox}

    \chapter{Построение обратной матрицы с помощью присоединенной матрицы.}

    \noindent Алгоритм построения обратной матрицы:
    \begin{enumerate}
        \item Получение союзной матрицы (матрицы алгебраических дополнений)
        \item Транспонирование союзной матрицы и получение присоединенной
        \item Деление каждого элементаприсоединенной матрицы на определитель исходной
    \end{enumerate}

    $$
    A^{-1}=\frac{1}{\det A}\hat A
    $$

    \chapter{Построение обратной матрицы с помощью элементарных преобразований.}

    Метод состоит в присоединении единичной матрицы к исходной справа и дальнейшем ее приведении к единичной. После этих операций справа появится обратная матрица.

    \chapter{Два определения ранга матрицы: максимальное число линейно независимых строк и размерность наибольшего ненулевого минора.}

    \begin{defbox}
        \textbf{Ранг матрицы} -- количество ее ненулевых строк, полученных после приведения к ступенчатому виду.
    \end{defbox}

    \begin{defbox}
        \textbf{Ранг матрицы} -- это наибольший порядок минора, отличного от нуля (теорема о базисном миноре)
    \end{defbox}

    \chapter{Линейная зависимость и независимость строк (столбцов) матрицы. Понятие базисного минора. Теорема о базисном миноре.}

    \begin{defbox}
        Набор элементов называется \textbf{линейно зависисым}, если равенство нулю их линейной комбинации возможно хотя бы при одном коэффициенте, не равном нулю.
        $$
        \exists i : \alpha_i \ne 0 \Rightarrow \alpha_1a_1 + \alpha_2a_2 + ... + \alpha_ia_i +...\alpha_na_n=0
        $$
        Если один из векторов $a_i$ нулевой, то набор линейно зависим.
    \end{defbox}

    \begin{defbox}
        \textbf{Базисный минор} -- это любой ненулевой минор наибольшего порядка
    \end{defbox}
    \vspace{10pt}
    \newpage
    \begin{thbox}
        {[Теорема о базисном миноре]}

        Строки (столбцы) матрицы, содержащие строки (столбцы) базисного минора линейно независимы. Любая строка (столбец) является линейной комбинацией базисных строк
    \end{thbox}

    \begin{prbox}
        {[Линейная независимость базисных строк (столбцов)]}

        Пусть базисный минор $M_{j_1...j_r}^{i_1...i_r}=M_r$ (располагается в $i_1...i_r$ строках и $j_1...j_r$ столбцах). Если столбцы матрицы, содержащие столбца базисного минора, линейно зависимы, то $\exists \lambda_1...\lambda_r : \forall i \Rightarrow \lambda_i \ne 0$:
        $$
        \lambda_1 a_{i1} +...+ \lambda_r a_{ir}=0
        $$
        $$
        \lambda_1 \begin{pmatrix}
            a_{1{j1}}\\
            ...\\
            a_{m{j1}}
        \end{pmatrix}
        +
        \dots
        + \lambda_r
        \begin{pmatrix}
            a_{1{jr}}\\
            ...\\
            a_{m{jr}}
        \end{pmatrix}
        =0
        $$

        Тогда выберем из полученной суммы строки базисного минора и получим противоречие:

        $$
        \lambda_1 \begin{pmatrix}
                a_{i1{j1}}\\
                ...\\
                a_{ir{j1}}
            \end{pmatrix}
            +
            \dots
            + \lambda_r
            \begin{pmatrix}
                a_{i1{jr}}\\
                ...\\
                a_{i1{jr}}
            \end{pmatrix}
            =0
    $$
    Это означает, что строки базисного минора линейно зависимы. Получаем противоречие изначальному предположению ($\det M = 0$).

    \vspace{20pt}
    {[Строка (столбец) как линейная комбинация базисных]}

    Рассмотрим $a_k \notin \lbrace a_j1, a_j2 ...a_jr\rbrace$ (к минору допишем столбец и строку из исходной матрицы)
    $$
    \Delta_p=
    \begin{pmatrix}
        a_{i1j1} & ... & a_{i1jr} & a_{i1k}\\
        ...\\
        a_{irj1} & ... & a_{irjr} & a_{irk}\\
        a_{pj1} & ... & a_{pjr} & a_{pk}
    \end{pmatrix}
    $$
    Если $p \in \lbrace i_1, i_2 ... i_r\rbrace$, то в матрице две строки одинаковые, и $\det \Delta_p = 0$

    Иначе $\Delta_p$ -- минор $r+1$ порядка $\Rightarrow \det \Delta_p=0$

    Разложим по строке $p$:

    $$
    \det \Delta_p = a_{pj1} A_{pj1} + ... + a_{pjr} A_{pjr} + a_{pk} A_{pk}=0
    $$
    $$
    A_{pk} = (-1)^{r+1+r+1} M_r = M_r \ne 0
    $$
    $$
    a_{pk} = -\frac{A_{r+1,1}}{M_r}-...-\frac{A_{r+1,r}}{M_r}a_{pjr}
    $$
    Но $\Delta_{r+1,r}$ не зависит от $p$. Тогда $\forall p -\frac{A_{r+1,i}}{M_r}=\lambda_i$. Распишем полученное для разных $p$:

    $a_{1k}=\lambda_1a_{1j1} + ... + \lambda_ra_{1jr}$

    $a_{2k}=\lambda_1a_{2j1} + ... + \lambda_ra_{2jr}$

    ...

    $a_{mk}=\lambda_1a_{mj1} + ... + \lambda_ra_{mjr}$

    То есть получили разложение $k$- го столбца по остальным. То есть разложение существует

    \end{prbox}

    \chapter{Вычисление ранга матрицы с помощью элементарных преобразований.}

    Суть метода заключается в приведении матрицы к ступенчатому виду, после чего просто останется посчитать количество ненулевых строк

    \chapter{Вычисление ранга матрицы: метод окаймляющих миноров.}

    Суть метода заключается в последовательном поиске ненулевых миноров, окаймляющих данный ненулевой минор. 
    
    \noindent Алгоритм:
    \begin{enumerate}
        \item Находим любой ненулевой элемент матрицы (минор 1-го порядка). 
        \item Находим ненулевой минор 2-го порядка, который содержит (окаймляет) наш ненулевой минор 1-го порядка. 
        \item Далее находим ненулевой минор 3-го порядка, который содержит (окаймляет) найденный ненулевой минор 2-го порядка. Если он есть, $\operatorname{rang} A \ge 3$.
        \item Продолжаем процесс, пока на каком-то шаге ($k+1$) все окаймляющие миноры для ненулевого минора $k$-го порядка не окажутся равными нулю. Тогда ранг матрицы равен $k$.
    \end{enumerate}

    \chapter{Теорема об инвариантности ранга матрицы при элементарных преобразованиях.}

    \begin{thbox}
        Ранг матрицы не меняется при элементарных преобразованиях её строк (или столбцов).
    \end{thbox}
    \begin{prbox}
        При умножении строки на число $\lambda \ne 0$ базисный минор умножится на $\lambda$. Ни один минор, равный нулю, не сделается отличным от нуля, и ни один минор, неравный нулю, не сделается нулевым.

        Теперь рассмотрим перестановку двух строк местами. Ранг матрицы будет равен размерности линейной оболочки ее строк: 
        
        $$
        \operatorname{rang} A = \dim L(a_1, a_2 ... a_n)
        $$

        где $a_i$ -- строки матрицы.

        Если переставить любые две строки матрицы местами, то размерность линейной оболочки не изменится, ведь не изменится ее базис. А поскольку не изменится размерность линейной оболочки, то не поменяется и ранг.
    \end{prbox}

    \chapter{Доказательство эквивалентности двух определений ранга.}
    \begin{thbox}
        Определение ранга матрицы через ненулевые строки (столбцы) эквивалентно определению через максимальный ненулевой минор
    \end{thbox}

    \begin{prbox}
        Пусть $r_r$ -- ранг по строкам, $r_m$ -- ранг по минорам.

        Докажем, что $r_r \le r_m$ и $r_r \ge r_m$

        Пусть существует ненулевой минор порядка $k$. Тогда строки этого минора линейно независимы, а значит линейно независимы и строки исходной матрицы, содержащей эти строки. Поэтому $k \le r_r$, а значит $r_m \le r_r$.

        Если в матрице можно выбрать $k$ линейно независимых строк, то существует как минимум ненулевой минор порядка $k$. Значит $r_m \ge k$, и поэтому $r_m \ge r_r$
        
        $$
        \begin{cases}
            r_m \ge r_r\\
            r_m \le r_r
        \end{cases}
        \Leftrightarrow
        r_m = r_r
        $$
    \end{prbox}

    \chapter{Теорема Кронекера-Капелли. Условие разрешимости неоднородной системы линейных уравнений}
    \begin{thbox}
        {[Теорема Кронекера-Капелли]}

        Система линейных уравнений имеет решения тогда и только тогда, когда когда ранг матрицы равен рангу расширенной матрицы системы
    \end{thbox}

    \begin{prbox}
        {[Совместна $\Rightarrow$ $\operatorname{rang} A = \operatorname{rang} A | b$]}

        Пусть $\exists x_0 \in \mathbb{R} : Ax_0 = b$. Тогда столбец $b$ -- линейная комбинация столбцов $A$. Следовательно, при добавлении к матрице $A$ столбца $b$ размерность пространства столбцов не изменится. То есть:
        $$
        \operatorname{rang} A = \operatorname{rang} A | b
        $$

        {[$\operatorname{rang} A = \operatorname{rang} A | b \Rightarrow$ Совместна]}

        Если $\operatorname{rang} A = \operatorname{rang} A | b$, то это означает, что не изменяется размерность пространства столбцов, то есть $b$ есть линейная комбинация столбцов матрицы $A$. То есть:
        
        $$
        \exists x_0 \in \mathbb{R} : Ax_0 = b
        $$

    \end{prbox}

    \chapter{Число решений неоднородной системы линейных уравнений. Свободные и зависимые переменные.}

    \section{Число решений неоднородной СЛАУ}

    \subsection{Случай 1: $r < n$}
    \begin{itemize}
        \item Существуют свободные переменные
        \item Система имеет бесконечность решений, зависящих от $n - r$ базисных переменных
    \end{itemize}

    \subsection{Случай 2: $r = n$}
    Система имеет единственное решение

    \subsection{Случай 3: $\operatorname{rang} A \ne \operatorname{rang} A|b$}
    Система не имеет решений

    \section{Свободные и зависимые переменные}
    \begin{defbox}
        \textbf{Свободные переменные} -- это переменные, которые выбираются в качестве параметров, от которых зависят другие переменные.
    \end{defbox}

    \begin{defbox}
        \textbf{Зависимые переменные }-- это переменные, значения которых зависят от свободных переменных
    \end{defbox}

    \chapter{Решение неоднородной системы линейных уравнений с помощью обратной матрицы.}
    Пусть есть СЛАУ:
    $$
    Ax=b
    $$

    Для ее решения умножим слева обе части равенства на матрицу $A^{-1}$:

    $$
    x = A^{-1}b
    $$

    Получим в соответствующих строках значения переменных.

    \chapter{Решение неоднородной системы линейных уравнений методом Крамера.}

    \begin{thbox}
        {[Метод Крамера]}

        Пусть есть СЛАУ:
        $$
        \begin{pmatrix}
            a_{11} & a_{12} & ... & a_{1n}\\
            ...\\
            a_{n1} & a_{n2} & ... & a_{nn}
        \end{pmatrix}
        \begin{pmatrix}
            x_1\\
            ...\\
            x_n
        \end{pmatrix}
        =
        \begin{pmatrix}
            b_1\\
            ...\\
            b_n
        \end{pmatrix}
        $$

        Решение СЛАУ можно получить следующим образом:

        $ x_1 = \frac{
        \begin{vmatrix}
            b_1 & a_{12} & ... & a_{1n}\\
            \dots\\
            b_n & a_{n2} & ... & a_{nn}
        \end{vmatrix}
        }{\det A}
        $, 
        $
        x_2 = \frac{
        \begin{vmatrix}
            a_{11} & b_1 & ... & a_{1n}\\
            \dots\\
            a_{n1} & b_n & ... & a_{nn}
        \end{vmatrix}
        }{\det A}
        $,
        $
        x_3 = \frac{
        \begin{vmatrix}
            a_{11} & a_{12} & ... & b_1\\
            \dots\\
            a_{n1} & a_{n2} & ... & b_n
        \end{vmatrix}
        }{\det A}
        $
    \end{thbox}

    \begin{prbox}
        Посмотрим на $A^{-1}b$:
        $$
        \frac{1}{\det A} \begin{pmatrix}
            A_{11} & A_{21} & ... & A_{n1}\\
            ...\\
            A_{1n} & A_{2n} & ... & A_{nn}
        \end{pmatrix}
        \begin{pmatrix}
            b_1\\
            ...\\
            b_n
        \end{pmatrix}
        =
        $$
        $$
        =\frac{1}{\det A}\begin{pmatrix}
            A_{11}b_1 + A_{21} b_2 + ... + A_{n1}b_n\\
            ...\\
            A_{1n}b_1 + A_{2n} b_2 + ... + A_{nn}b_n
        \end{pmatrix}
        $$

        Отсюда видно, что строки последней матрицы представляют собой определители матриц $A$, в которых на место одного ее столбца подставлен столбец $b$.

    \end{prbox}

    \chapter{Решение неоднородной системы линейных уравнений методом Гаусса. Ступенчатая (трапециевидная) матрица системы уравнений.}

    Метод заключается в приведении расширенной матрицы к ступенчатому виду путем элементарных преобразований (они не меняют решения исходной системы) с последующим обратным ходом, либо продолжением элементарных преобразований до приведения матрицы к диагональному виду.

    \begin{defbox}
        Матрица называется \textbf{ступенчатой}, если:
        \begin{enumerate}
            \item Все нулевые строки стоят ниже нулевых
            \item Номера столбцов ведущих элементов строк образуют возрастающую последовательность
        \end{enumerate}
    \end{defbox}

    \chapter{Решение однородной системы линейных уравнений методом Гаусса}
    Метод заключается в приведении матрицы к ступенчатому виду путем элементарных преобразований (они не меняют решения исходной системы) с последующим обратным ходом, либо продолжением элементарных преобразований до приведения матрицы к диагональному виду.

    \chapter{Фундаментальное множество решений однородной системы линейных уравнений. Структура множества решений неоднородной системы линейных уравнений.}
    \begin{defbox}
        \textbf{Фундаментальной системой решений (ФСР)} однородной системы $Ax=0$ называется базис её пространства решений, то есть набор из $n-r$ линейно независимых решений, через которые любое решение системы выражается как их линейная комбинация.
    \end{defbox}

    Структура решений неоднородной СЛАУ задается следующим образом:
    $$
    x_\text{общ} = x_\text{част} + x_\text{одн}
    $$
    Где:
    \begin{enumerate}
        \item $x_\text{част}$ -- любое частное решение системы
        \item $x_\text{одн}$ -- все решения соответствующей однородной системы $Ax=0$, которые выражаются через ФСР
    \end{enumerate}

    \chapter{Определение линейного (векторного) пространства.}

    \begin{defbox}
        \textbf{Векторным пространством} называется множество, над элементами которого определены две операции: сложение и умножение на действительное число, удовлетворяющее следующим аксиомам:
        \begin{enumerate}
            \item $a+b=b+a$
            \item $(a+b)+c=a+(b+c)$
            \item $\exists 0: \forall a \Rightarrow a + 0 = a$
            \item $\exists 1: \forall a \Rightarrow a \cdot 1 = a$
            \item $\forall a \exists \widetilde{a} : a +\widetilde{a} = 0$
            \item $\alpha(\beta\gamma) = \alpha\beta\gamma$
            \item $(\alpha+\beta)a = \alpha a + \beta a$
            \item $\alpha (a+b) = \alpha a + \alpha b$
        \end{enumerate}
    \end{defbox}

    \chapter{Определение линейной комбинации векторов. Линейно зависимые и линейно независимые системы векторов.}

    \begin{defbox}
        Пусть $a_1, ... a_n$ -- элементы линейного пространства. \textbf{Линейной комбинацией} называетсявыражение вида:
        $$
        \alpha_1 a_1 + \alpha_2 a_2 + ... + \alpha_n a_n
        $$
        Где $\alpha_i \in \mathbb{R}$
    \end{defbox}

    \begin{defbox}
        Набор элементов называется \textbf{линейно зависимым}, если равенство нулю их линейной комбинации возможно хотя бы при одном коэффициенте, не равном нулю
        $$
        \exists \alpha_i \ne 0 : \alpha_1 a_1 + \alpha_2 a_2 + ... + \alpha_n a_n = 0
        $$
        Если один из  векторов $a_i$ нулевой, то набор линейно зависим
    \end{defbox}

    \chapter{Базис линейного пространства. Размерность пространства.}
    \begin{defbox}
        {[Определение №1]}

        \textbf{Базисом} называется такой набор ненулевых линейно независимых векторов, что любой вектор линейного пространства может быть разложен по этому набору
        
        \vspace{20pt}
        {[Определение №2]}

        \textbf{Базисом} называется линейно независимый набор векторов такой, что добавление к этому набору любого вектора пространства делает расширенный набор линейно зависимым

        \vspace{20pt}
        {[Определение №3]}

        \textbf{Базисом} линейного пространства называется максимально возможный по числу векторов линейно независимый набор векторов пространства
    \end{defbox}

    \begin{defbox}
        \textbf{Размерностью} линейного пространства называется количество векторов в базисе
    \end{defbox}

    \chapter{Разложение вектора по базису. Координаты вектора в базисе.}
    \begin{thbox}
        Разложение по базису единственно.
    \end{thbox}

    \begin{prbox}
        Пусть это не так и есть два разложения вектора $x$ в базисе $e_1, e_2 ... e_n$:
        $$
        x = x_1e_1 + x_2e_2 + ... + x_ne_n = y_1e_1 + y_2e_2 + ... + y_ne_n
        $$
        Найдется $i : x_i \ne y_i$
        $$
        (x_1-y_1)e_1 + (x_2-y_2)e_2 + ... + (x_n-y_n)e_n = 0
        $$
        Но тогда выходит, что линейная комбинация базисных векторов равна нулю, а значит базисные векторы линейно зависимы. Противоречие.
    \end{prbox}

    \begin{defbox}
        \textbf{Координаты вектора в базисе} -- это набор коэффициентов, на которые нужно умножить базисные векторы для получение исходного вектора
    \end{defbox}

    \chapter{Переход к новому базису в линейного пространстве. Матрица перехода. Формулы преобразований координат вектора.}
    Пусть есть старый базис: $e_1, ... e_n$. Разложим новый базис $e'_1, ... e'_n$ по старому базису:
    $$
    \begin{cases}
        e'_1 = c_{11}e_1 + c_{12}e_2 + ... + c_{1n}e_n\\
        e'_2 = c_{21}e_1 + c_{22}e_2 + ... + c_{2n}e_n\\
        \dots\\
        e'_n = c_{n1}e_1 + c_{n2}e_2 + ... + c_{nn}e_n
    \end{cases}    
    $$
    $$
    \begin{pmatrix}
        e'_1\\
        e'_2\\
        \dots\\
        e'_n
    \end{pmatrix}
    =
    \begin{pmatrix}
        c_{11} & c_{12} & ... & c_{1n}\\
        c_{21} & c_{22} & ... & c_{2n}\\
        ...\\
        c_{n1} & c_{n2} & ... & c_{nn}\\
    \end{pmatrix}
    \begin{pmatrix}
        e_1\\
        e_2\\
        ...\\
        e_n
    \end{pmatrix}
    =
    C^T\begin{pmatrix}
        e_1\\
        e_2\\
        ...\\
        e_n
    \end{pmatrix}
    $$
    Где $C^T$ -- матрица координат, записанных столбцами, нового базиса в старом

    \begin{defbox}
        \textbf{Матрица перехода} -- матрица, в которой столбцами записаны координаты нового базиса в старом. При ее умножении на столбец старого базиса получается новый базис
    \end{defbox}

    Таким образом можно легко переходить между базисами. $C^T$ обратима, поскольку новый базис, как и старый, линейно независим. Значит матрица перехода может также использоваться для перехода в обратную сторону: от нового базиса к старому:
    $$
    \begin{pmatrix}
        e'_1\\
        e'_2\\
        \dots\\
        e'_n
    \end{pmatrix}
    =
    C^T\begin{pmatrix}
        e_1\\
        e_2\\
        ...\\
        e_n
    \end{pmatrix}
    \Leftrightarrow
    (C^{-1})^T
    \begin{pmatrix}
        e'_1\\
        e'_2\\
        \dots\\
        e'_n
    \end{pmatrix}
    =
    \begin{pmatrix}
        e_1\\
        e_2\\
        ...\\
        e_n
    \end{pmatrix}
    $$

    \chapter{Линейное отображение линейных пространств. Пример: проектирование.}

    \begin{defbox}
        Пусть $V$ и $W$ -- линейные пространства над одним и тем же полем $\mathbb{R}$. Отображение $f : V \rightarrow W$ называется \textbf{линейным}, если $\forall u, v \in V, \alpha \in \mathbb{R}$ выполняются условия:
        \begin{enumerate}
            \item $f(u+v) = f(u) +f(v)$
            \item $f(\alpha u) = \alpha f(u)$
        \end{enumerate}
    \end{defbox}

    Проверим такое отображение $\operatorname{Pr}_x : \mathbb{R}^2 \rightarrow \mathbb{R}^2$, которое каждому вектору в $\mathbb{R}^2$ ставит в соответствие его проекцию на ось $Ox$. То есть вектор $\operatorname{Pr}_x((x, y)) = (x, 0)$
    \begin{enumerate}
        \item Возьмем $u = (x_1, y_1)$ и $v = (x_2, y_2)$. 
        $$
        \operatorname{Pr}_x(u+v) = \operatorname{Pr}_x((x_1+x_2, y_1+y_2))=(x_1+x_2, 0)
        $$
        $$
        \operatorname{Pr}_x(u) + \operatorname{Pr}_x(v) = (x_1, 0) + (x_2, 0) = (x_1+x_2, 0)
        $$
        Линейность сохраняется
        \item Возьмем скаляр $\alpha$ и вектор $u=(x, y)$
        $$
        \operatorname{Pr}_x(\alpha u) = \operatorname{Pr}_x(\alpha x, \alpha y) = (\alpha x, 0)
        $$
        $$
        \alpha \operatorname{Pr}_x(u) = \alpha (x, 0) = (\alpha x, 0)
        $$
        Умножение на скаляр сохраняется
    \end{enumerate}

    \chapter{Изоморфизм линейных пространств. Пространство вектор-столбцов $\mathbb{R}^n$. Базисные векторы.}
    \begin{defbox}
        Пусть есть пространства $L_1$, $L_2$. \textbf{Изоморфизмом} этих пространств называется биекция, отображающая $L_1$ в $L_2$ такая, что:
        \begin{enumerate}
            \item $\forall (x, y) \Rightarrow f(u+v) = f(u) +f(v)$
            \item $\forall \alpha \in \mathbb{R} \Rightarrow f(\alpha u) = \alpha f(u)$
        \end{enumerate}
    \end{defbox}

    \begin{defbox}
        $\mathbb{R}^n$ — это множество всех упорядоченных наборов из n вещественных чисел, записанных в виде столбца:
        $$
        \begin{pmatrix}
            a_1\\
            a_2\\
            \dots\\
            a_n
        \end{pmatrix}
        $$
        где $a_i \in \mathbb{R}$
    \end{defbox}
    \begin{defbox}
        \textbf{Базисом} линейного пространства называется максимально возможный по числу векторов линейно независимый набор векторов пространства
    \end{defbox}

    \chapter{Теорема о равномощности базисов. Размерность линейного пространства.}

    \begin{thbox}
        Любые два базиса линейного пространства состоят из одного и того же количества векторов
    \end{thbox}

    \begin{prbox}
        Допустим, это не так и в линейном пространстве $L$ есть два неравномощных базиса: $e_1, ... e_n$ и $f_1, ... f_{n+1}$

        Выразим базис $f$ через базис $e$:
        $$
        \begin{cases}
            f_1 = c_{11}e_1 + c_{12}e_2 + ... + c_{1n}e_n\\
        f_2 = c_{21}e_1 + c_{22}e_2 + ... + c_{2n}e_n\\
        \dots\\
        f_n = c_{n1}e_1 + c_{n2}e_2 + ... + c_{nn}e_n\\
        f_{n+1} = c_{{n+1},1}e_1 + c_{{n+1},2}e_2 + ... + c_{{n+1},n}e_n\\
        \end{cases}
        $$
        
        Рассмотрим линейную комбинацию:
        $$
        x_1f_1+x_2f_2+...+x_{n+1}f_{n+1}=0
        $$
        Если окажется, что такая система имеет решение при $x_i \ne 0$, то докажем противоречие

        $$
        x_1(c_{11}e_1 + c_{12}e_2 + ... + c_{1n}e_n)+
        $$
        $$
        +x_2(c_{21}e_1 + c_{22}e_2 + ... + c_{2n}e_n)+
        $$
        $$
        +...+x_{n+1}(c_{{n+1},1}e_1 + c_{{n+1},2}e_2 + ... + c_{{n+1},n}e_n)=0
        $$

        $$
        e_1(x_1c_{11}+x_2c_{21}+...+x_{n+1}c_{n+1,1})+
        $$
        $$
        +e_2(x_1c_{12}+x_2c_{22}+...+x_{n+1}c_{n+1,2})+
        $$
        $$
        +...+e_{n}(x_1c_{1n}+x_2c_{2n}+...+x_{n+1}c_{n+1,n})=0
        $$

        Поскольку $e_1...e_n$ -- базис, то коэффициенты в линейной комбинации равны нулю:
        $$
        \begin{cases}
            x_1c_{11}+x_2c_{21}+...+x_{n+1}c_{n+1,1}=0\\
            x_1c_{12}+x_2c_{22}+...+x_{n+1}c_{n+1,2}=0\\
            \dots\\
            x_1c_{1n}+x_2c_{2n}+...+x_{n+1}c_{n+1,n}=0
        \end{cases}
        $$
        Эта система имеет $n$ уравнений и $n+1$ переменных, а значит она имеет бесконечное множество решений, то есть существуют такие $x_1,...x_{n+1}$, что $x_1f_1+x_2f_2+...+x_{n+1}f_{n+1}=0$, а значит $f_1, ... f_{n+1}$ -- не базис!

    \end{prbox}

    \chapter{Определение линейной оболочки системы векторов.}
    \begin{defbox}
        Пусть есть набор $u_1,...u_n$ векторов в $L$.

        \textbf{Линейная оболочка} системы векторов -- это множество всех линейных комбинация этих векторов
        $$
        L<u_1\dots u_n>=\lbrace\alpha_1u_1 + \alpha_2u_2 + \dots +\alpha_nu_n : \alpha_1, \dots \alpha_n \in \mathbb{R}\rbrace
        $$
    \end{defbox}

    \chapter{Определение подпространства линейного пространства. Базис подпространства.}
    \begin{defbox}
        Пусть $L$ -- линейное пространство над некоторым полем.

        Пусть $L_1$ -- подмножество векторов (элементов) $L$ такое, что выполняются:
        \begin{enumerate}
            \item $x \in L_1 \wedge y \in L_1 \Leftrightarrow x + y \in L_1$
            \item $x \in L_1 \Leftrightarrow \forall \alpha \Rightarrow \alpha x \in L_1$
        \end{enumerate}
        Тогда $L_1$ называется \textbf{подпространством} $L$
    \end{defbox}
    \begin{thbox}
        Если $L_1$ -- собственное подпространство $L$, то $\dim L > \dim L_1$
    \end{thbox}
    \begin{prbox}
        Пусть $\dim L = \dim L_1$. Базис $L_1$: $e_1,\dots e_n$

        Разложим некоторый вектор $x \in L_1$ бо базису $L_1$:
        $$
        x = x_1e_1 + \dots + x_ne_n
        $$
        Но $\dim L = n$, при этом по аксиоме $x \in L$. Значит $e_1,\dots e_n$ -- базис $L$. То есть любой вектор из $L_1$ содержится и в $L$, а значит $L=L_1$. Противоречие 
    \end{prbox}

    \chapter{Способы задания подпространства. Линейная оболочка векторов и пространство решений однородной системы линейных уравнений.}

    Подпространство можно задать двумя способами: линейной оболочкой системы векторов и множеством решений однородной системы уравнений 
    $$
    L=<u_1,...u_n>
    $$
    или
    $$
    Ax=0
    $$
    Рассмотрим подробнее способ задания подпространства через однородную СЛАУ:
    \begin{enumerate}
        \item $n=r$: решение единственное, $x=\lbrace 0 \rbrace$, нулевое подпространство
        \item $n>r$: бесконечное множество решений
    \end{enumerate}

    \chapter{Нахождение базиса подпространства, порожденного системой векторов.}
    Пусть задана система векторов: $u_1, \dots u_n$. Их линейная оболочка: $L=<u_1,\dots u_n>$.
    Чтобы найти базис такого подпространства, необходимо проверить систему векторов на линейную зависимость, после чего взять из них только линейно независимые. Это и будет базис.

    \chapter{Сумма подпространств. Теорема Грассмана о размерности суммы подпространств}
    \begin{defbox}
        Пусть $U$ и $V$ -- подпространства в $L$. \textbf{Суммой} подпространств называется множество тех тех элементов $L$, которые удовлетворяют условию:
        $$
        z \in L \Rightarrow z = x + y
        $$
        Где $x \in U$, $y \in V$
    \end{defbox}

    \begin{thbox}
        {[Теорема Грассмана]}

        Сумма размерностей подпространств равна сумме размерностей их пересечения и суммы
        $$
        \dim U_1 + \dim U_2 = \dim U_1+U_2 -\dim U_1 \cap U_2
        $$
    \end{thbox}
    \begin{prbox}
        Базис пересечения: $h_1,\dots h_k$

        Векторы базиса $U_1$, не входящие в пересечение: $f_1,\dots f_l$

        Векторы базиса $U_2$, не входящие в пересечение: $g_1,\dots g_m$

        Базис $U_1$: $h_1,\dots h_k, f_1, \dots f_l$

        Базис $U_2$: $h_1, \dots h_k, g_1, \dots g_m$

        Докажем, что $h_1,\dots h_k, f_1, \dots f_l, g_1, \dots g_m$ -- базис (пусть нет).
        $$
        \lambda_1 h_1 + \dots + \lambda_k h_k + \mu_1 f_1 + \dots + \mu_l f_l + \nu_1 g_1 + \dots + \nu_m g_m = 0
        $$
        $$
        \lambda_1 h_1 + \dots + \lambda_k h_k + \mu_1 f_1 + \dots + \mu_l f_l = -\nu_1 g_1 - \dots - \nu_m g_m 
        $$
        Причем правая часть лежит в $U_1 \cap U_2$, а значит правая тоже. Это означает, что $\lambda_1=\dots\lambda_k=0$, но тогда $f_1, \dots f_l$ входит в пересечение. Противоречие

        Посчитаем размерность:
        $$
        \dim U_1 + \dim U_2 = k + l + k + m=2k + l + m
        $$
        $$
        \dim U_1 + U_2 = k + l + m 
        $$
        Значит разность: $k = \dim U_1 \cap U_2$

        Итого:
        $$
        \boxed{\dim U_1 + \dim U_2 = \dim U_1+U_2 -\dim U_1 \cap U_2}
        $$
    \end{prbox}
\end{document}