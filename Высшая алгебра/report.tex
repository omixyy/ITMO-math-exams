\documentclass[14pt]{extreport}
\usepackage{graphicx}
\usepackage{gost}
\usepackage[T1,T2A]{fontenc}
\usepackage[utf8]{inputenc}
\PassOptionsToPackage{russian,english}{babel}
\usepackage[english,russian]{babel}
\usepackage{tempora}
\usepackage{hyperref}
\linespread{1.3}
\setlength{\parindent}{1.25cm}
\usepackage{hyperref}

\usepackage{multirow}
\usepackage{makecell}
\usepackage{longtable}
\usepackage{array}
\newcolumntype{P}[1]{>{\raggedright\arraybackslash}p{#1}}

\usepackage{fancyhdr}
\pagestyle{fancy}
\fancyhf{} 
\fancyhead{} 
\fancyfoot[C]{\thepage} 

\usepackage{amsmath, amssymb}
\usepackage{tcolorbox}
\usepackage{enumitem}
\usepackage{geometry}
\usepackage{amsthm}

\newtcolorbox{thbox}[1][]{colback=orange!10, colframe=orange!60, title=Теорема, #1}
\newtcolorbox{prbox}[1][]{colback=green!5, colframe=green!60!gray, title=Доказательство, #1}
\newtcolorbox{defbox}[1][]{colback=blue!5, colframe=blue!60, title=Определение, #1}
\newtcolorbox{impbox}[1][]{colback=red!5, colframe=red!60, title=Важно, #1}

\newcommand{\defeq}[0]{\stackrel{\text{def}}{=}}

\begin{document}
\thispagestyle{empty}
	\title{Высшая алгебра. Экзамен.}
	\maketitle

	\newpage
	\tableofcontents

    \newpage
    \chapter{Матрица. Определение сложения матриц, умножения матрицы на число, свойства}
    \section{Матрица}

    \begin{defbox}
        \textbf{Матрица} -- прямоугольная таблица чисел $m \times n$ 
    \end{defbox}

    \section{Действия над матрицами}

    \begin{defbox}
        \textbf{Сложение матриц} -- операция над матрицами, при которой складываются их соответственные элементы. Для матриц разных размеров операция не определена.
        $$
        \begin{pmatrix}
            1 & 0 & 0\\
            0 & 1 & 1\\
            1 & 1 & 0
        \end{pmatrix}
        +
        \begin{pmatrix}
            0 & 1 & 1\\
            1 & 0 & 0\\
            0 & 0 & 1
        \end{pmatrix}
        =\begin{pmatrix}
            1 & 1 & 1\\
            1 & 1 & 1\\
            1 & 1 & 1
        \end{pmatrix}
        $$
    \end{defbox}

    \begin{defbox}
        \textbf{Умножение матрицы на число} -- операция, при которой каждый элемент матрицы умножается на данное число.

        $$
        \lambda \begin{pmatrix}
            1 & 1\\
            0 & 1
        \end{pmatrix}
        =
        \begin{pmatrix}
            \lambda & \lambda \\
            0 & \lambda
        \end{pmatrix}
        $$
    \end{defbox}

    \chapter{Определение и свойства операции умножения матриц}
    \section{Умножение матриц}
    \begin{defbox}
        \textbf{Умножение матрицы на матрицу} -- это операция над двумя матрицами, при которой каждый $i$-й элемент $n$-й строки первой матрицы умножается на $i$-ый элемент $n$-го столбца второй матрицы, после чего все такие произведения суммируются
        $$
        A \cdot B = C
        $$, где 
        $$
        c_{ij} = \sum_{t=1}^{k}a_{it}b_{tj}
        $$
    \end{defbox}
    \section{Свойства умножения матриц}
    \begin{enumerate}
        \item $A(BC) = (AB)C$
        \item $A(B+C) = AB+AC$
        \item $(B+C)D = BD + CD$
    \end{enumerate}

    \begin{impbox}
        Умножение матриц не коммутативно!
        $$
        AB \ne BA
        $$
    \end{impbox}

    \chapter{Транспонирование матрицы. Свойства}

    \begin{defbox}
        \textbf{Транспонирование матрицы} -- процесс преобразование матрицы, при котором столбцы становятся строками, а строки -- столбцами.
        $$
        A = \begin{pmatrix}
            0 & 1 & 1\\
            1 & 0 & 0\\
            0 & 0 & 1
        \end{pmatrix}
        \Leftrightarrow
        A^T = \begin{pmatrix}
            0 & 1 & 0\\
            1 & 0 & 0\\
            1 & 0 & 1
        \end{pmatrix}
        $$
    \end{defbox}

    \noindent Свойства транспонирования матрицы:
    \begin{enumerate}
        \item $(A^T)^T=A$
        \item $A^T+B^T=(A+B)^T$
        \item $\lambda A^T = (\lambda A)^T$
        \item $(AB)^T=B^TA^T$
        \item $(A^{-1})^T= (A^T)^{-1}$
        \item $\det A = \det A^T$
        \item $\operatorname{rang} A = \operatorname{rang} A^T$
    \end{enumerate}

    \chapter{Отдельные виды матриц: квадратная, треугольная, диагональная, единичная. Определения и свойства}

    \section{Виды матриц}

    \begin{defbox}
        Матрица называется \textbf{квадратной}, если число ее строк равно числу ее столбцов
        $$
        A = \begin{pmatrix}
            0 & 1 & 0\\
            1 & 0 & 0\\
            1 & 0 & 1
        \end{pmatrix}
        $$
    \end{defbox}

    \noindent Свойства квадратной матрицы:
    \begin{enumerate}
        \item Можно вычислить определитель
        \item Может быть обратимой, если $\det A \ne 0$
        \item Можно возводить в целую степень
    \end{enumerate}
    \newpage

    \begin{defbox}
        Матрица называется \textbf{верхнетреугольной}, если она является квадратной, а \textbf{под} ее главной диагональю все элементы равны нулю
        $$
        A = \begin{pmatrix}
            0 & 1 & 2\\
            0 & 3 & 1\\
            0 & 0 & 1
        \end{pmatrix}
        $$
    \end{defbox}

    \begin{defbox}
        Матрица называется \textbf{нижнетреугольной}, если она является квадратной, а \textbf{над} ее главной диагональю все элементы равны нулю
        $$
        A = \begin{pmatrix}
            0 & 0 & 0\\
            2 & 3 & 0\\
            1 & 0 & 1
        \end{pmatrix}
        $$
    \end{defbox}

    \noindent Свойства треугольных матриц:
    \begin{enumerate}
        \item Определитель равен произведению диагональных элементов
        \item Обратная к треугольной матрице (если существует) тоже треугольная того же типа
        \item Произведение двух верхних (нижних) треугольных матриц — верхняя (нижняя) треугольная матрица
    \end{enumerate}

    \newpage

    \begin{defbox}
        Матрица называется \textbf{диагональной}, если все ее элементы, кроме тех, что лежат на главной диагонали, равны нулю
        $$
        A = \begin{pmatrix}
            2 & 0 & 0\\
            0 & 3 & 0\\
            0 & 0 & 1
        \end{pmatrix}
        $$
    \end{defbox}
    \noindent Свойства диагональной матрицы:
    \begin{enumerate}
        \item Определитель равен произведению диагональных элементов
        \item Обратная: $D^{-1}=\operatorname{diag} \left( d_1^{-1}, d_2^{-1}, ...\right)$
        \item Возведение в степень: $D^k=\operatorname{diag} \left( d_1^k, d_2^k, ...\right)$
    \end{enumerate}
    \begin{defbox}
        Матрица называется \textbf{единичной}, если все элементы ее главной диагонали равны единице, а остальные -- нулю
        $$
        E = \begin{pmatrix}
            1 & 0 & 0\\
            0 & 1 & 0\\
            0 & 0 & 1
        \end{pmatrix}
        $$
    \end{defbox}
    \noindent Свойства единичной матрицы:
    \begin{enumerate}
        \item Нейтральный элемент относительно умножения
        \item $\det E = 1$
        \item $E^{-1}=E$
    \end{enumerate}

\end{document}